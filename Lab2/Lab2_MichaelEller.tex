% !TEX TS-program = pdflatex
% !TEX encoding = UTF-8 Unicode

% This is a simple template for a LaTeX document using the "article" class.
% See "book", "report", "letter" for other types of document.

\documentclass{article} % use larger type; default would be 10pt

\usepackage[utf8]{inputenc} % set input encoding (not needed with XeLaTeX)

%%% Examples of Article customizations
% These packages are optional, depending whether you want the features they provide.
% See the LaTeX Companion or other references for full information.

%%% PAGE DIMENSIONS
\usepackage{geometry} % to change the page dimensions
\geometry{letterpaper} % or letterpaper (US) or a5paper or....
 \geometry{margin=1in} % for example, change the margins to 2 inches all round
% \geometry{landscape} % set up the page for landscape
%   read geometry.pdf for detailed page layout information

\usepackage{graphicx} % support the \includegraphics command and options
\usepackage[normalem]{ulem}
% \usepackage[parfill]{parskip} % Activate to begin paragraphs with an empty line rather than an indent
%\usepackage[subsection]{placeins}
\usepackage{float}
\usepackage [autostyle, english = american]{csquotes}
%%% PACKAGES
\usepackage{booktabs} % for much better looking tables
\usepackage{array} % for better arrays (eg matrices) in maths
\usepackage{paralist} % very flexible & customisable lists (eg. enumerate/itemize, etc.)
\usepackage{verbatim} % adds environment for commenting out blocks of text & for better verbatim
\usepackage{subfig} % make it possible to include more than one captioned figure/table in a single float
% These packages are all incorporated in the memoir class to one degree or another...
\usepackage{mcode}
\usepackage{lstlinebgrd}
%%% HEADERS & FOOTERS
\usepackage{fancyhdr} % This should be set AFTER setting up the page geometry
\pagestyle{fancy} % options: empty , plain , fancy
\renewcommand{\headrulewidth}{0pt} % customise the layout...
\lhead{}\chead{}\rhead{}
\lfoot{}\cfoot{\thepage}\rfoot{}

%%% SECTION TITLE APPEARANCE
\usepackage{sectsty}
\allsectionsfont{\sffamily\mdseries\upshape} % (See the fntguide.pdf for font help)
% (This matches ConTeXt defaults)

%%% ToC (table of contents) APPEARANCE
\usepackage[nottoc,notlof,notlot]{tocbibind} % Put the bibliography in the ToC
\usepackage[titles,subfigure]{tocloft} % Alter the style of the Table of Contents
\renewcommand{\cftsecfont}{\rmfamily\mdseries\upshape}
\renewcommand{\cftsecpagefont}{\rmfamily\mdseries\upshape} % No bold!
\renewcommand{\topfraction}{0.85}
\renewcommand{\textfraction}{0.1}
\renewcommand{\floatpagefraction}{0.85}

%%% END Article customizations
\usepackage{float}
\usepackage{framed}
\usepackage{amsfonts}
\usepackage{amsmath}
\usepackage{mathabx}
%%% The "real" document content comes below...
\newcommand{\includecode}[2][c]{\lstinputlisting[caption=#2, escapechar=, style=custom#1]{#2}<!---->}
\title{Lab 2}
\author{Michael Eller}
\date{February 15, 2016} % Activate to display a given date or no date (if empty),
         	      % otherwise the current date is printed 

\begin{document}
\maketitle

\tableofcontents
\clearpage
\section{Introduction}
In the first lab, we explored some of the many techniques used in audio industry to organize and search through music by content. We developed identifiers to
attach to the music such as its loudness and frequency distribution. In this lab, we will be classifying tracks according to the identifiers developed in the first lab. 
Since many of the calculations were operating independently for each frame, parallelizing the process using MATLAB's \mcode{parfor} function, we are able
to divide the processes unto multiple threads. Personally, this reduced my computation times by 50\%, despite parallelization overhead. 

\begin{framed}
\textbf{Assignment}
\begin{enumerate}
\item Modify your function that computes the \textbf{mfcc} to return the \textbf{mfcc} in decibels. 
\item Construct a two-dimensional \emph{spectrum histogram} (a matrix of counts), with 40 columns - one for each mfcc index, and 50 rows - one
for each amplitude level, measured in dB, from -20 to 60 dB. You will normalize the histogram, such that the sum along each column (for each "note") is one.
\item Display, using the MATLAB function \textbf{imagesec}, the spectrum histogram for the 12 audio tracks supplied for the first lab. 
\end{enumerate}
\end{framed}

\section{Rhythm}
Rhythm can be defined as the presence of repetitive patterns of movement or sound. In this lab, we will be using several techniques to detect the presence
of rhythm in an audio track.

\subsection{Spectrum Histogram}
Extending from Lab1, we slightly modify our function that computes the \textbf{mfcc} coefficients in order to add a non-linearity. We will, however, only be using
this nonlinear \textbf{mfcc} function in calculating the spectrum histograms. 

\lstinputlisting[caption=mfcc.m,language=MATLAB,numbers=left]{./mfcc.m}

In order to better visualize the \textbf{mfcc} output, we create a \emph{spectrum histogram} which maps each mfcc index and the corresponding amplitude levels
to a color displaying how often each mfcc index occured at the respective volume. For example, in Figure \ref{fig:hist201_1}, aside from the very low sounds, most
of the pitches were at around the same volume. It is interesting to note in Figure \ref{fig:hist370} in Appendix \ref{sec:specHis}, 
there is a high concentration at the lower left corner. This makes sense though.The song has practically no bass to it, and the Spectrum Histogram reflects that. 
There should also exist a high similarity along the main diagonal, as a frame is completely similar to itself. 

\begin{figure}[H]
\centering
\includegraphics[width=\textwidth]{specHistogram201-classical.png}
\caption{Spectrum Histogram 201}
\label{fig:hist201_1}
\end{figure}

\subsection{Similarity Matrix}
\begin{framed}
\textbf{Assignment}
\begin{enumerate}
\setcounter{enumi}{3}
\item Implement the computation of the similarity matrix, and display the similarity matrices (color-coded) for the 12 audio tracks supplied for the first lab. 
\end{enumerate}
\end{framed}

The accompanying code and figure for the similarity matrix for "track201-classical.wav" can be seen below. The remaining figures can be found in Appendix \ref{sec:Sim}.

\lstinputlisting[caption=simMatrix.m,language=MATLAB,numbers=left]{./simMatrix.m}

\begin{figure}[H]
\centering
\includegraphics[width=\textwidth]{SimMatrix201-classical.png}
\caption{Similarity Matrix 201}
\label{fig:sim201}
\end{figure}

The first stage involves computing a similarity matrix defined as follows

\begin{equation}
\label{eq:sim}
S(i,j) = \frac{\langle \text{mfcc}(:,i), \text{mfcc}(:,j) \rangle }{||\text{mfcc}(:,i) || \,  ||\text{mfcc}(:,j) ||} = 
\sum\limits_{k=1}^{\text{nbanks}} \frac{\text{mfcc}(k,i)\  \text{mfcc}(k,j)}{||\text{mfcc}(:,i) || \,  ||\text{mfcc}(:,j) ||}
\end{equation}

Currently the similarity matrices are not quite correct. While they do possess all the correct features in that they are symmetrical, their outputs do not make sense. 
According to my function, every frame appears extremely similar. With an ideal function, one would be able to see \emph{hotspots} on the map, corresponding to
longer notes, with the size of each hotspot being inversely proportional to the speed of the rhythmic progression. Using Figure \ref{fig:sim201} as an example, 
it would imply that near the beginning of the piece there is a general rhythmic pattern. But as the piece progresses past the 300\textsuperscript{th} frame, the similarity
drastically decreases. The musical equivalent of this would be the piece has sped up insanely fast and now there is a different pitch almost every $\frac{1}{f_s\times 255}$
(~0.0231) seconds. This would imply that there were up to 43 new pitches every second. 

\subsection{A first Estimate of the Rhythm}
Once we note that the entries in our similarity matrix are always at a distance of l frames, we are able to better quantify the rhythm by creating a rhythm index, defined
by Equation \ref{eq:rhy}.

\begin{equation}
\label{eq:rhy}
B(l)=\frac{1}{N_f-l} \sum\limits_{n=1}^{N_f-l}  S(n,n+l), \quad l=0,\dots N_f -1
\end{equation}

\begin{framed}
\textbf{Assignment}
\begin{enumerate}
\setcounter{enumi}{4}
\item Implement the computation of the rhythm index $B(l)$ defined in Equation \ref{eq:rhy}. Plot the vector B as a function of the lag $l=0,N_f -1$  for the 12 tracks. 
Comment on the presence, or absence, of a strong rhythmic pattern. \\
Hint: to debug your function, you can use track-396 which should display strong rhythmic structure. 
\end{enumerate}
\end{framed}

\begin{figure}[H]
\centering
\includegraphics[width=\textwidth]{RhythmIndex396-electronic.png}
\caption{Rhythm Index 396}
\label{fig:rhy396}
\end{figure}

As can be seen in Figure \ref{fig:rhy396}, There is a very periodic pattern throughout the song. The periodicity of the graph in Figure \ref{fig:rhy396} may hint
at the underlying subdivision of the beat. This makes sense for track 396, as it has a very repetitive thumping bass line throughout the song. The spikes on the left
side of the graph are more akin to faster notes while slower notes are represented further right. 

%--------------------------------------%
\subsection{A Better Estimate of the Rhythm}
Rather than compounding the similarity between frames that are at a fixed lag apart in time, we can try to detect more interesting patterns in the similarity matrix. 
In addition to knowing the similarity between two frames, we would also like to know if this similarity is repeated later in the segment, at time j+l. 
The autocorrelation of frames can be defined as:

\begin{equation}
\label{eq:autoc}
AR(l)= \frac{1}{N_f(N_f-l)} \sum\limits_{i=1}^{N_f}  \sum\limits_{j=1}^{N_f-l} S(i,j)S(i,j+l), \quad l=0,\dots,N_f-1
\end{equation}

\begin{framed}
\textbf{Assignment}
\begin{enumerate}
\setcounter{enumi}{5}
\item Implement the computation of the rhythm index $AR(l)$ defined in Equation \ref{eq:autoc}. Plot the vector $AR$ as a function of the lag $l=0,N_f-1$ for the 12 tracks.
Comment on the presence, or absence, of a strong rhythmic pattern.
\end{enumerate}
\end{framed}

Equation \ref{eq:autoc} can be represented in MATLAB with Listing  \ref{ls:autoC} below. 

\lstinputlisting[caption=AutoC.m,language=MATLAB,numbers=left,label=ls:autoC]{./AutoC.m} 

The plot for the autocorrelation of track396-electronic.wav can be seen in Figure \ref{fig:auto396} below. The plot of the autocorrelation is strikingly similar to the plots for the 
rhythm index, B(l). At this point in analysis, the better choice would be whichever method is faster. 
The remaining figures can be found in Appendix \ref{sec:autocfig}. 

\begin{figure}[H]
\centering
\includegraphics[width=\textwidth]{AutoC396-electronic.png}
\caption{Autocorrelation 396}
\label{fig:auto396}
\end{figure}

%--------------------------------------%
\subsection{Rhythmic Variations over Time}

Here we are interested in studying the dynamic changes in the rhythm. For this purpose, we consider short times windows formed by 20 frames (approximately 1 second)
over which we compute a vector $AR$ of size 20. 

\begin{equation}
\label{eq:ARLm}
\begin{aligned}
AR(l,m)= \frac{1}{20(20-l)} \sum\limits_{i=1}^{20}  \sum\limits_{j=1}^{20-l} S(i+m*20,j+m*20)S(i+m*20,j+m*20+l)\\
\text{for }  l=0,\dots,19, \quad \text{and } m=0,\dots,N_f/20-1
\end{aligned}
\end{equation}


\begin{framed}
\textbf{Assignment}
\begin{enumerate}
\setcounter{enumi}{6}
\item Implement the computation of the rhythm index $AR(l,m)$ defined in Equation \ref{eq:ARLm}. Plot the image $AR$ in false colors as a
function of the lag $l=0,\dots,19$ (y-axis) and the time window index $m$ (x-axis) for the 12 tracks. Comment on the rhythm patterns for the different tracks. 
\end{enumerate}
\end{framed}


%--------------------------------------%

\clearpage
\appendix
%--------------------------------------%
\section{Figures}
\subsection{Spectrum Histogram}
\label{sec:specHis}

\begin{figure}[H]
\centering
\includegraphics[width=0.75\textwidth]{specHistogram201-classical.png}
\caption{Spectrum Histogram 201}
\label{fig:hist201}
\end{figure}

\begin{figure}[H]
\centering
\includegraphics[width=0.75\textwidth]{specHistogram204-classical.png}
\caption{Spectrum Histogram 204}
\label{fig:hist204}
\end{figure}


\begin{figure}[H]
\centering
\includegraphics[width=0.75\textwidth]{specHistogram370-electronic.png}
\caption{Spectrum Histogram 370}
\label{fig:hist370}
\end{figure}

\begin{figure}[H]
\centering
\includegraphics[width=0.75\textwidth]{specHistogram396-electronic.png}
\caption{Spectrum Histogram 396}
\label{fig:hist396}
\end{figure}

\begin{figure}[H]
\centering
\includegraphics[width=0.75\textwidth]{specHistogram437-jazz.png}
\caption{Spectrum Histogram 437}
\label{fig:hist437}
\end{figure}

\begin{figure}[H]
\centering
\includegraphics[width=0.75\textwidth]{specHistogram439-jazz.png}
\caption{Spectrum Histogram 439}
\label{fig:hist439}
\end{figure}

\begin{figure}[H]
\centering
\includegraphics[width=0.75\textwidth]{specHistogram463-metal.png}
\caption{Spectrum Histogram 463}
\label{fig:hist463}
\end{figure}

\begin{figure}[H]
\centering
\includegraphics[width=0.75\textwidth]{specHistogram492-metal.png}
\caption{Spectrum Histogram 492}
\label{fig:hist492}
\end{figure}

\begin{figure}[H]
\centering
\includegraphics[width=0.75\textwidth]{specHistogram547-rock.png}
\caption{Spectrum Histogram 547}
\label{fig:hist547}
\end{figure}

\begin{figure}[H]
\centering
\includegraphics[width=0.75\textwidth]{specHistogram550-rock.png}
\caption{Spectrum Histogram 550}
\label{fig:hist550}
\end{figure}

\begin{figure}[H]
\centering
\includegraphics[width=0.75\textwidth]{specHistogram707-world.png}
\caption{Spectrum Histogram 707}
\label{fig:hist707}
\end{figure}

\begin{figure}[H]
\centering
\includegraphics[width=0.75\textwidth]{specHistogram729-world.png}
\caption{Spectrum Histogram 729}
\label{fig:hist729}
\end{figure}
\clearpage





%--------------------------------------%

\subsection{Similarity Matrix}
\label{sec:Sim}


\begin{figure}[H]
\centering
\includegraphics[width=0.75\textwidth]{SimMatrix201-classical.png}
\caption{Similarity Matrix 201}
\label{fig:sim201_1}
\end{figure}

\begin{figure}[H]
\centering
\includegraphics[width=0.75\textwidth]{SimMatrix204-classical.png}
\caption{Similarity Matrix 204}
\label{fig:sim204_1}
\end{figure}


\begin{figure}[H]
\centering
\includegraphics[width=0.75\textwidth]{SimMatrix370-electronic.png}
\caption{Similarity Matrix 370}
\label{fig:sim370_1}
\end{figure}

\begin{figure}[H]
\centering
\includegraphics[width=0.75\textwidth]{SimMatrix396-electronic.png}
\caption{Similarity Matrix 396}
\label{fig:sim396_1}
\end{figure}

\begin{figure}[H]
\centering
\includegraphics[width=0.75\textwidth]{SimMatrix437-jazz.png}
\caption{Similarity Matrix 437}
\label{fig:sim437_1}
\end{figure}

\begin{figure}[H]
\centering
\includegraphics[width=0.75\textwidth]{SimMatrix439-jazz.png}
\caption{Similarity Matrix 439}
\label{fig:sim439_1}
\end{figure}

\begin{figure}[H]
\centering
\includegraphics[width=0.75\textwidth]{SimMatrix463-metal.png}
\caption{Similarity Matrix 463}
\label{fig:sim463_1}
\end{figure}

\begin{figure}[H]
\centering
\includegraphics[width=0.75\textwidth]{SimMatrix492-metal.png}
\caption{Similarity Matrix 492}
\label{fig:sim492_1}
\end{figure}

\begin{figure}[H]
\centering
\includegraphics[width=0.75\textwidth]{SimMatrix547-rock.png}
\caption{Similarity Matrix 547}
\label{fig:sim547_1}
\end{figure}

\begin{figure}[H]
\centering
\includegraphics[width=0.75\textwidth]{SimMatrix550-rock.png}
\caption{Similarity Matrix 550}
\label{fig:sim550_1}
\end{figure}

\begin{figure}[H]
\centering
\includegraphics[width=0.75\textwidth]{SimMatrix707-world.png}
\caption{Similarity Matrix 707}
\label{fig:sim707_1}
\end{figure}

\begin{figure}[H]
\centering
\includegraphics[width=0.75\textwidth]{SimMatrix729-world.png}
\caption{Similarity Matrix 729}
\label{fig:sim729_1}
\end{figure}





%--------------------------------------%

\subsection{Rhythm Index}
\label{sec:rhythmFig}


\begin{figure}[H]
\centering
\includegraphics[width=0.75\textwidth]{RhythmIndex201-classical.png}
\caption{Rhythm Index 201}
\label{fig:rhy201_1}
\end{figure}

\begin{figure}[H]
\centering
\includegraphics[width=0.75\textwidth]{RhythmIndex204-classical.png}
\caption{Rhythm Index 204}
\label{fig:rhy204_1}
\end{figure}


\begin{figure}[H]
\centering
\includegraphics[width=0.75\textwidth]{RhythmIndex370-electronic.png}
\caption{Rhythm Index 370}
\label{fig:rhy370_1}
\end{figure}

\begin{figure}[H]
\centering
\includegraphics[width=0.75\textwidth]{RhythmIndex396-electronic.png}
\caption{Rhythm Index 396}
\label{fig:rhy396_1}
\end{figure}

\begin{figure}[H]
\centering
\includegraphics[width=0.75\textwidth]{RhythmIndex437-jazz.png}
\caption{Rhythm Index 437}
\label{fig:rhy437_1}
\end{figure}

\begin{figure}[H]
\centering
\includegraphics[width=0.75\textwidth]{RhythmIndex439-jazz.png}
\caption{Rhythm Index 439}
\label{fig:rhy439_1}
\end{figure}

\begin{figure}[H]
\centering
\includegraphics[width=0.75\textwidth]{RhythmIndex463-metal.png}
\caption{Rhythm Index 463}
\label{fig:rhy463_1}
\end{figure}

\begin{figure}[H]
\centering
\includegraphics[width=0.75\textwidth]{RhythmIndex492-metal.png}
\caption{Rhythm Index 492}
\label{fig:rhy492_1}
\end{figure}

\begin{figure}[H]
\centering
\includegraphics[width=0.75\textwidth]{RhythmIndex547-rock.png}
\caption{Rhythm Index 547}
\label{fig:rhy547_1}
\end{figure}

\begin{figure}[H]
\centering
\includegraphics[width=0.75\textwidth]{RhythmIndex550-rock.png}
\caption{Rhythm Index 550}
\label{fig:rhy550_1}
\end{figure}

\begin{figure}[H]
\centering
\includegraphics[width=0.75\textwidth]{RhythmIndex707-world.png}
\caption{Rhythm Index 707}
\label{fig:rhy707_1}
\end{figure}

\begin{figure}[H]
\centering
\includegraphics[width=0.75\textwidth]{RhythmIndex729-world.png}
\caption{Rhythm Index 729}
\label{fig:rhy729_1}
\end{figure}
\clearpage



%--------------------------------------%
\subsection{Autocorrelation}
\label{sec:autocfig}

\begin{figure}[H]
\centering
\includegraphics[width=0.75\textwidth]{AutoC201-classical.png}
\caption{Autocorrelation 201}
\label{fig:auto201_1}
\end{figure}

\begin{figure}[H]
\centering
\includegraphics[width=0.75\textwidth]{AutoC204-classical.png}
\caption{Autocorrelation 204}
\label{fig:auto204_1}
\end{figure}


\begin{figure}[H]
\centering
\includegraphics[width=0.75\textwidth]{AutoC370-electronic.png}
\caption{Autocorrelation 370}
\label{fig:auto370_1}
\end{figure}

\begin{figure}[H]
\centering
\includegraphics[width=0.75\textwidth]{AutoC396-electronic.png}
\caption{Autocorrelation 396}
\label{fig:auto396_1}
\end{figure}

\begin{figure}[H]
\centering
\includegraphics[width=0.75\textwidth]{AutoC437-jazz.png}
\caption{Autocorrelation 437}
\label{fig:auto437_1}
\end{figure}

\begin{figure}[H]
\centering
\includegraphics[width=0.75\textwidth]{AutoC439-jazz.png}
\caption{Autocorrelation 439}
\label{fig:auto439_1}
\end{figure}

\begin{figure}[H]
\centering
\includegraphics[width=0.75\textwidth]{AutoC463-metal.png}
\caption{Autocorrelation 463}
\label{fig:auto463_1}
\end{figure}

\begin{figure}[H]
\centering
\includegraphics[width=0.75\textwidth]{AutoC492-metal.png}
\caption{Autocorrelation 492}
\label{fig:auto492_1}
\end{figure}

\begin{figure}[H]
\centering
\includegraphics[width=0.75\textwidth]{AutoC547-rock.png}
\caption{Autocorrelation 547}
\label{fig:auto547_1}
\end{figure}

\begin{figure}[H]
\centering
\includegraphics[width=0.75\textwidth]{AutoC550-rock.png}
\caption{Autocorrelation 550}
\label{fig:auto550_1}
\end{figure}

\begin{figure}[H]
\centering
\includegraphics[width=0.75\textwidth]{AutoC707-world.png}
\caption{Autocorrelation 707}
\label{fig:auto707_1}
\end{figure}

\begin{figure}[H]
\centering
\includegraphics[width=0.75\textwidth]{AutoC729-world.png}
\caption{Autocorrelation 729}
\label{fig:auto729_1}
\end{figure}
\clearpage



%--------------------------------------%
\clearpage
\section{Code}

\lstinputlisting[caption=mfcc.m,language=MATLAB,numbers=left]{./mfcc.m}
\lstinputlisting[caption=spectrumHistogram.m,language=MATLAB,numbers=left]{./spectrumHistogram.m}
\lstinputlisting[caption=simMatrix.m,language=MATLAB,numbers=left]{./simMatrix.m}
\lstinputlisting[caption=rhythmIndex.m,language=MATLAB,numbers=left]{./rhythmIndex.m}
\lstinputlisting[caption=autoC.m,language=MATLAB,numbers=left]{./autoC.m}











\lstinputlisting[caption=buildReport.m,language=MATLAB,numbers=left]{./buildReport.m}

\end{document}