% !TEX TS-program = pdflatex
% !TEX encoding = UTF-8 Unicode

% This is a simple template for a LaTeX document using the "article" class.
% See "book", "report", "letter" for other types of document.

\documentclass{article} % use larger type; default would be 10pt

\usepackage[utf8]{inputenc} % set input encoding (not needed with XeLaTeX)

%%% Examples of Article customizations
% These packages are optional, depending whether you want the features they provide.
% See the LaTeX Companion or other references for full information.

%%% PAGE DIMENSIONS
\usepackage{geometry} % to change the page dimensions
\geometry{letterpaper} % or letterpaper (US) or a5paper or....
 \geometry{margin=1in} % for example, change the margins to 2 inches all round
% \geometry{landscape} % set up the page for landscape
%   read geometry.pdf for detailed page layout information
\usepackage{lipsum}
%\usepackage{cases}
\usepackage{graphicx} % support the \includegraphics command and options
\usepackage{subcaption}
\usepackage[normalem]{ulem}
% \usepackage[parfill]{parskip} % Activate to begin paragraphs with an empty line rather than an indent
%\usepackage[subsection]{placeins}
\usepackage{float}
\usepackage [autostyle, english = american]{csquotes}
%%% PACKAGES
\usepackage{booktabs} % for much better looking tables
\usepackage{array} % for better arrays (eg matrices) in maths
\usepackage{paralist} % very flexible & customisable lists (eg. enumerate/itemize, etc.)
\usepackage{verbatim} % adds environment for commenting out blocks of text & for better verbatim
\usepackage{subfig} % make it possible to include more than one captioned figure/table in a single float
% These packages are all incorporated in the memoir class to one degree or another...
\usepackage{mcode}
\usepackage{xfrac}
\usepackage{lstlinebgrd}
%%% HEADERS & FOOTERS
\usepackage{fancyhdr} % This should be set AFTER setting up the page geometry
\pagestyle{fancy} % options: empty , plain , fancy
\renewcommand{\headrulewidth}{0pt} % customise the layout...
\lhead{}\chead{}\rhead{}
\lfoot{}\cfoot{\thepage}\rfoot{}
\usepackage{mathtools}
%%% SECTION TITLE APPEARANCE
\usepackage{sectsty}
\allsectionsfont{\sffamily\mdseries\upshape} % (See the fntguide.pdf for font help)
% (This matches ConTeXt defaults)

%%% ToC (table of contents) APPEARANCE
\usepackage[nottoc,notlof,notlot]{tocbibind} % Put the bibliography in the ToC
\usepackage[titles,subfigure]{tocloft} % Alter the style of the Table of Contents
\renewcommand{\cftsecfont}{\rmfamily\mdseries\upshape}
\renewcommand{\cftsecpagefont}{\rmfamily\mdseries\upshape} % No bold!
\renewcommand{\topfraction}{0.85}
\renewcommand{\textfraction}{0.1}
\renewcommand{\floatpagefraction}{0.85}

%%% END Article customizations
\usepackage{float}
\usepackage{framed}
\usepackage{amsfonts}
\usepackage{amsmath}
\usepackage{mathabx}
%%% The "real" document content comes below...
\newcommand{\includecode}[2][c]{\lstinputlisting[caption=#2, escapechar=, style=custom#1]{#2}<!---->}
\title{Lab 4}
\author{Michael Eller}
\date{March 28, 2016} % Activate to display a given date or no date (if empty),
         	      % otherwise the current date is printed 

\begin{document}
\maketitle

\tableofcontents
\clearpage
\section{Introduction}

During this lab, we will be investigating the implementation of Layer III
of MPEGG 1, also known as mp3. We will fist develop several subband
filters to decompose and reconstruct the original audio signal. We will
be using a polyphase pseudo Quadrature Mirror Filter to deconstruct
and eventually reconstruct the original audio signal. 

\section{Cosine Modulated Pseudo Quadrature Mirror Filter: Analysis}

In this section, we will manipulate the equations that mathematically
describe the analysis filter. \\


Consider the filter $h_K$, then the output of the combined filtering by $h_k$ and decimation is given by

\begin{equation}
\label{eq:skn}
s_k[n]=\sum\limits_{m=0}^{511} h_k[m]x[32n-m]
\end{equation}

where

\begin{equation}
\label{eq:hkn2}
h_k[n]=p_n[n] \cos \left( \frac{(2k+1)(r-16)\pi}{64}\right)
\quad k=0,\dots,31, \; n=0,\dots,511
\end{equation}

and $p_0$ is a prototype lowpass filter. The role of $h_k$ is clear: the
modulation of $p_0$ by $\cos \left( \frac{(2k+1)(r-16)\pi}{64}\right)$ shifts
the lowpass filter around frequency $(2k+1)\pi/64$. 
Equation \ref{eq:skn} requires $32 \times 512 = 16,384$ combined multiplications
and additions to compute the 32 outputs $s_{1},\dots,s_{32}$ for each block
of 32 samples of the incoming signal. This is simply too slow to be properly
effective. 



We define:

\begin{equation}
\label{eq:cn}
c[n]=
\begin{dcases*}
-p_0[n] & if $[n/64]$ is odd\\
+p_0[n]  & otherwise
\end{dcases*}
\end{equation}


then

\begin{equation}
\label{eq:hk}
h_k[64q+r]=c[64q+r] \, \cos \left( \frac{(2k+1)(r-16)\pi}{64}\right)
\end{equation}

Using the notations of the standard, we further define


\begin{equation}
\label{eq:mkr}
\boxed{
M_{k,r}=\cos \left( \frac{(2k+1)(r-16)\pi}{64}\right), 
\quad k=0,\dots,31, \; r=0,\dots,63
}
\end{equation}

then

\begin{equation}
h_k[64q+r]=c[64q+r]M_{k,r}
\end{equation}

and


\begin{align}
s_k[n]&=\sum\limits_{r=0}^{63} \sum\limits_{q=0}^{7} 
c[64q+r] M_{k,r} x[32n-64q-r]\\ 
& =  \sum\limits_{r=0}^{63} M_{k,r} \sum\limits_{q=0}^{7} 
c[64q+r] x[32n-64q-r]
\end{align}


In summary, for every integer $m=32n$, multiple of 32, the convolution from
equation \ref{eq:skn} can be quickly computed using the following three steps,

\begin{align}
z[64q+r] &= c[64q+r]x[m-64q-r], &r&=0,\dots,63, \; q=0,\dots,7
\label{eq:zqr}\\
y[r]	 &= \sum \limits _{q=0}^{7} z[64q+r], &r&=0,\dots,63 
\label{eq:yr}\\
s[k]	 &= \sum \limits _{r=0} ^{63} M_{k,r}\; y[r],  &k&=0,\dots,31
 \label{eq:sk2}
\end{align}

Even further speedup can be obtained by using a fast DCT algorithm to compute
the matrix-vector multiplication in equation \ref{eq:sk2}.

%\begin{equation}
%\label{eq:KL}
%KL(G^s,G^{s'})=\frac{1}{2}\left (\text{tr}(\Sigma_{s'}^{-1}\Sigma_s) + (\mu_{s'}-\mu_s)^T \Sigma_{s'}^{-1} (\mu_{s'}-\mu_s)-K+
%\log\left( \frac{\det \Sigma_{s'}}{\det \Sigma_s} \right) \right )
%\end{equation}





\begin{framed}
\textbf{Assignment}
\begin{enumerate}
\item Write the MATLAB \verb|pqmf| that implements the analysis filter bank
described in equations \ref{eq:mkr}-\ref{eq:zqr}. The function will have the following template:
\\[2ex]
\verb|[coefficients] = pqmf (input)|
\\[2ex]
where \verb|input| is a buffer that contains an integer number of frames
of audio data. The output array \verb|coefficients| has the same
size as the buffer \verb|input|, and contains the subband coefficients. 
\\

The array \verb|coefficients| should be organized in the following manner:

\begin{equation}
\texttt{coefficients} = 
\left[ S_{0}[0] \; \dots \; S_{0}[N_S-1] \; 
\dots S_{31}[0] \; \dots \; S_{31}[N_S-1]\right]
\end{equation}

where $S_i[k]$ is the coefficient from subband $i=0,\dots,31$ computed
for the packet $k$ of 32 audio samples. Also $N_S$ is the total number of
packets of 32 samples:

\begin{equation}
N_S=\frac{\texttt{Samples}}{32}=18*\texttt{nFrames}
\end{equation}

The organization of \verb|coefficients| is such that the low frequencies
come first, and then the next higher frequencies, and so on and so forth. 

\item Analyse the first 5 seconds of the following tracks, and display the
array \verb|coefficients|,

\begin{itemize}
\item sample1.wav, sample2.wav
\item sine1.wav, sine2.wav
\item handel.wav
\item cast.wav
\item gilberto.wav
\end{itemize}

Comment on the visual content of the arrays \verb|coefficients|.
\end{enumerate}
\end{framed}

%%%%%%%%%%%%%%%%%%%%%%%%%%%%%%%%%%%%%%%% Figures Question 2

\begin{figure}[H]
%\begin{subfigure}{0.5\textwidth}
\centering
\includegraphics[width=\textwidth]{sample1_5sec.png}
\caption{PQFM: Sample1 5 seconds}
\label{fig:samp1PQFM}
\end{figure}
\begin{figure}[H]
\centering
\includegraphics[width=\textwidth]{sample2_5sec.png}
\caption{PQFM: Sample2 5 seconds}
\label{fig:samp2PQFM}
\end{figure}
%\caption{Samples Files}
%\end{figure}

\begin{figure}[H]

\begin{subfigure}{0.5\textwidth}
\centering
\includegraphics[width=\textwidth]{sine1_1sec.png}
\caption{PQFM: Sine1}
\label{fig:sin1PQFM}
\end{subfigure}
\begin{subfigure}{0.5\textwidth}
\centering
\includegraphics[width=\textwidth]{sine2_2sec.png}
\caption{PQFM: Sine2}
\label{fig:sin2PQFM}
\end{subfigure}
\caption{Sine Waves}
\end{figure}


\begin{figure}[H]
\centering
\includegraphics[width=0.8\textwidth]{handel_5sec.png}
\caption{PQFM: Handel 5 seconds}
\label{fig:handPQFM}
\end{figure}

\begin{figure}[H]
\centering
\includegraphics[width=0.8\textwidth]{cast_5sec.png}
\caption{PQFM: Castanets 5 seconds}
\label{fig:castPQFM}
\end{figure}


\begin{figure}[H]
\centering
\includegraphics[width=0.8\textwidth]{gilberto_5sec.png}
\caption{PQFM: Gilberto 5 seconds}
\label{fig:gilbPQFM}
\end{figure}

%\lipsum{1}

The PQFM coefficients essentially show an outline of the frequency spectrums
of the various songs. 

As seen in Figure \ref{fig:samp1PQFM}, the extreme lower end of the PQFM
coefficients are low, while they get larger as you approach coefficient number
5000. If one listens to the song, it can be easily heard how the piano lacks
a substantial bass, but the higher notes can be heard more easily. 

Looking at Figure \ref{fig:samp2PQFM}, the lower end of the spectrum is much
more substantial, as can be heard by the thumping bass in the song. Since the
music is mostly electronic, it lacks many of the stronger overtones that are
commonly found on natural classical instruments. As you go even higher in the
spectrum, the coefficients do not approach zero as quickly as was the case
in Figure \ref{fig:samp1PQFM}. Perhaps this is due to the inevitable 
higher-order harmonics that result from electronic music. 

Figures \ref{fig:sin1PQFM} and \ref{fig:sin2PQFM} show the most discernible
distinctions. The first sine wave is obviously lower than the second because
of the larger concentration around the lower end of the spectrum. 

Figure \ref{fig:handPQFM} is about the same as Figure \ref{fig:samp1PQFM}. There
is more activity on the lower end of the spectrum though. This is probably due
to the fact that "handel.wav" includes vocals as well as piano and violin. 


Figure \ref{fig:castPQFM} is quite interesting. While the previous figures had
a fairly consistent decline as the coefficients increased, this trend
seems fairly haphazard. Something of considerable note though is that
this is the first plot to have neighbouring coefficients of unequal magnitude.
While the other plots were extremely symmetrical about the 0 amplitude point, 
Figure \ref{fig:castPQFM} is not. Perhaps this is due to the highly percussive
nature of the castanets and its atonal sounds. 

Gilberto


%%%%%%%%%%%%%%%%%%%%%%%%%%%%%%%%%%%%%%%%%%%%%%%%%%%%%%%














%\lstinputlisting[caption=kullbackDistance.m,language=MATLAB,numbers=left,label=ls:KLdm]{./kullbackDistance.m} 





%\begin{figure}[H]
%\begin{subfigure}{0.5\textwidth}
%\centering
%\includegraphics[width=\textwidth]{distanceMatrixChroma.png}
%\caption{Distance Matrix: Chroma}
%\label{fig:dMatChr}
%\end{subfigure}
%\begin{subfigure}{0.5\textwidth}
%\centering
%\includegraphics[width=\textwidth]{distanceMatrixMFCC.png}
%\caption{Distance Matrix: MFCC}
%\label{fig:dMatChr}
%\end{subfigure}
%\end{figure}



%\begin{figure}[H]
%\centering
%\includegraphics[width=\textwidth]{genreHistogram.png}
%\caption{Genre Histograms}
%\label{fig:genreHisto}
%\end{figure}



%%%%%%%%%%%%%%%%%%%%%%%%%%%%%%%%%%%%%%%%%%%%%%%%%%%%%%%%%%%%%
\clearpage
\appendix
\section{Figures}

\subsection{Distance Matrix}
\subsubsection{Chroma}


%\begin{figure}[H]
%\centering
%\includegraphics[width=0.75\textwidth]{distanceMatrixChroma30.png}
%\caption{Distance Matrix Chroma: 30 seconds}
%\end{figure}


%\lstinputlisting[caption=confuseMatrix.m,language=MATLAB,numbers=left]{./confuseMatrix.m} 


\end{document}