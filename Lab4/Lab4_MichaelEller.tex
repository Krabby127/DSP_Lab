% !TEX TS-program = pdflatex
% !TEX encoding = UTF-8 Unicode

% This is a simple template for a LaTeX document using the "article" class.
% See "book", "report", "letter" for other types of document.

\documentclass{article} % use larger type; default would be 10pt

\usepackage[utf8]{inputenc} % set input encoding (not needed with XeLaTeX)

%%% Examples of Article customizations
% These packages are optional, depending whether you want the features they provide.
% See the LaTeX Companion or other references for full information.

%%% PAGE DIMENSIONS
\usepackage{geometry} % to change the page dimensions
\geometry{letterpaper} % or letterpaper (US) or a5paper or....
 \geometry{margin=1in} % for example, change the margins to 2 inches all round
% \geometry{landscape} % set up the page for landscape
%   read geometry.pdf for detailed page layout information
\usepackage{lipsum}
%\usepackage{cases}
\usepackage{graphicx} % support the \includegraphics command and options
\usepackage{subcaption}
\usepackage[normalem]{ulem}
% \usepackage[parfill]{parskip} % Activate to begin paragraphs with an empty line rather than an indent
%\usepackage[subsection]{placeins}
\usepackage{float}
\usepackage [autostyle, english = american]{csquotes}
%%% PACKAGES
\usepackage{booktabs} % for much better looking tables
\usepackage{array} % for better arrays (eg matrices) in maths
\usepackage{paralist} % very flexible & customisable lists (eg. enumerate/itemize, etc.)
\usepackage{verbatim} % adds environment for commenting out blocks of text & for better verbatim
\usepackage{subfig} % make it possible to include more than one captioned figure/table in a single float
% These packages are all incorporated in the memoir class to one degree or another...
\usepackage{mcode}
\usepackage{xfrac}
\usepackage{lstlinebgrd}
%%% HEADERS & FOOTERS
\usepackage{fancyhdr} % This should be set AFTER setting up the page geometry
\pagestyle{fancy} % options: empty , plain , fancy
\renewcommand{\headrulewidth}{0pt} % customise the layout...
\lhead{}\chead{}\rhead{}
\lfoot{}\cfoot{\thepage}\rfoot{}
\usepackage{mathtools}
%%% SECTION TITLE APPEARANCE
\usepackage{sectsty}
\allsectionsfont{\sffamily\mdseries\upshape} % (See the fntguide.pdf for font help)
% (This matches ConTeXt defaults)

%%% ToC (table of contents) APPEARANCE
\usepackage[nottoc,notlof,notlot]{tocbibind} % Put the bibliography in the ToC
\usepackage[titles,subfigure]{tocloft} % Alter the style of the Table of Contents
\renewcommand{\cftsecfont}{\rmfamily\mdseries\upshape}
\renewcommand{\cftsecpagefont}{\rmfamily\mdseries\upshape} % No bold!
\renewcommand{\topfraction}{0.85}
\renewcommand{\textfraction}{0.1}
\renewcommand{\floatpagefraction}{0.85}

%%% END Article customizations
\usepackage{float}
\usepackage{framed}
\usepackage{amsfonts}
\usepackage{amsmath}
\usepackage{mathabx}
%%% The "real" document content comes below...
\newcommand{\includecode}[2][c]{\lstinputlisting[caption=#2, escapechar=, style=custom#1]{#2}<!---->}
\title{Lab 4}
\author{Michael Eller}
\date{March 28, 2016} % Activate to display a given date or no date (if empty),
         	      % otherwise the current date is printed 

\begin{document}
\maketitle

\tableofcontents
\clearpage
\section{Introduction}

During this lab, we will be investigating the implementation of Layer III
of MPEGG 1, also known as mp3. We will fist develop several subband
filters to decompose and reconstruct the original audio signal. We will
be using a polyphase pseudo Quadrature Mirror Filter to deconstruct
and eventually reconstruct the original audio signal. 

\section{Cosine Modulated Pseudo Quadrature Mirror Filter: Analysis}

In this section, we will manipulate the equations that mathematically
describe the analysis filter. \\


Consider the filter $h_K$, then the output of the combined filtering by $h_k$
and decimation is given by

\begin{equation}
\label{eq:skn}
s_k[n]=\sum\limits_{m=0}^{511} h_k[m]x[32n-m]
\end{equation}

where

\begin{equation}
\label{eq:hkn2}
h_k[n]=p_n[n] \cos \left( \frac{(2k+1)(r-16)\pi}{64}\right)
\quad k=0,\dots,31, \; n=0,\dots,511
\end{equation}

and $p_0$ is a prototype lowpass filter. The role of $h_k$ is clear: the
modulation of $p_0$ by $\cos \left( \frac{(2k+1)(r-16)\pi}{64}\right)$ shifts
the lowpass filter around frequency $(2k+1)\pi/64$. 
Equation \ref{eq:skn} requires $32 \times 512 = 16,384$ combined multiplications
and additions to compute the 32 outputs $s_{1},\dots,s_{32}$ for each block
of 32 samples of the incoming signal. This is simply too slow to be properly
effective. 



We define:

\begin{equation}
\label{eq:cn}
c[n]=
\begin{dcases*}
-p_0[n] & if $[n/64]$ is odd\\
+p_0[n]  & otherwise
\end{dcases*}
\end{equation}


then

\begin{equation}
\label{eq:hk}
h_k[64q+r]=c[64q+r] \, \cos \left( \frac{(2k+1)(r-16)\pi}{64}\right)
\end{equation}

Using the notations of the standard, we further define


\begin{equation}
\label{eq:mkr}
\boxed{
M_{k,r}=\cos \left( \frac{(2k+1)(r-16)\pi}{64}\right), 
\quad k=0,\dots,31, \; r=0,\dots,63
}
\end{equation}

then

\begin{equation}
h_k[64q+r]=c[64q+r]M_{k,r}
\end{equation}

and


\begin{align}
s_k[n]&=\sum\limits_{r=0}^{63} \sum\limits_{q=0}^{7} 
c[64q+r] M_{k,r} x[32n-64q-r]\\ 
& =  \sum\limits_{r=0}^{63} M_{k,r} \sum\limits_{q=0}^{7} 
c[64q+r] x[32n-64q-r]
\end{align}


In summary, for every integer $m=32n$, multiple of 32, the convolution from
equation \ref{eq:skn} can be quickly computed using the following three steps,

\begin{align}
z[64q+r] &= c[64q+r]x[m-64q-r], &r&=0,\dots,63, \; q=0,\dots,7
\label{eq:zqr}\\
y[r]	 &= \sum \limits _{q=0}^{7} z[64q+r], &r&=0,\dots,63 
\label{eq:yr}\\
s[k]	 &= \sum \limits _{r=0} ^{63} M_{k,r}\; y[r],  &k&=0,\dots,31
 \label{eq:sk2}
\end{align}

Even further speedup can be obtained by using a fast DCT algorithm to compute
the matrix-vector multiplication in equation \ref{eq:sk2}.

%\begin{equation}
%\label{eq:KL}
%KL(G^s,G^{s'})=\frac{1}{2}\left (\text{tr}(\Sigma_{s'}^{-1}\Sigma_s) + (\mu_{s'}-\mu_s)^T \Sigma_{s'}^{-1} (\mu_{s'}-\mu_s)-K+
%\log\left( \frac{\det \Sigma_{s'}}{\det \Sigma_s} \right) \right )
%\end{equation}





\begin{framed}
\textbf{Assignment}
\begin{enumerate}
\item Write the MATLAB \verb|pqmf| that implements the analysis filter bank
described in equations \ref{eq:mkr}-\ref{eq:zqr}. The function will have the
following template:
\\[2ex]
\verb|[coefficients] = pqmf (input)|
\\[2ex]
where \verb|input| is a buffer that contains an integer number of frames
of audio data. The output array \verb|coefficients| has the same
size as the buffer \verb|input|, and contains the subband coefficients. 
\\

The array \verb|coefficients| should be organized in the following manner:

\begin{equation}
\texttt{coefficients} = 
\left[ S_{0}[0] \; \dots \; S_{0}[N_S-1] \; 
\dots S_{31}[0] \; \dots \; S_{31}[N_S-1]\right]
\end{equation}

where $S_i[k]$ is the coefficient from subband $i=0,\dots,31$ computed
for the packet $k$ of 32 audio samples. Also $N_S$ is the total number of
packets of 32 samples:

\begin{equation}
N_S=\frac{\texttt{Samples}}{32}=18*\texttt{nFrames}
\end{equation}

The organization of \verb|coefficients| is such that the low frequencies
come first, and then the next higher frequencies, and so on and so forth. 

\item Analyse the first 5 seconds of the following tracks, and display the
array \verb|coefficients|,

\begin{itemize}
\item sample1.wav, sample2.wav
\item sine1.wav, sine2.wav
\item handel.wav
\item cast.wav
\item gilberto.wav
\end{itemize}

Comment on the visual content of the arrays \verb|coefficients|.
\end{enumerate}
\end{framed}

The MATLAB code used to implement the \verb|pqmf| function can be seen below
in Listing \ref{ls:pqmf}.  

\lstinputlisting[caption=pqmf.m,language=MATLAB,numbers=left,label=ls:pqmf]
{./pqmf.m} 

%%%%%%%%%%%%%%%%%%%%%%%%%%%%%%%%%%%%%%%% Figures Question 2

\begin{figure}[H]
\centering
\includegraphics[width=\textwidth]{sample1_5sec.png}
\caption{PQFM: Sample1 5 seconds}
\label{fig:samp1PQFM}
\end{figure}
\begin{figure}[H]
\centering
\includegraphics[width=\textwidth]{sample2_5sec.png}
\caption{PQFM: Sample2 5 seconds}
\label{fig:samp2PQFM}
\end{figure}


\begin{figure}[H]

\begin{subfigure}{0.5\textwidth}
\centering
\includegraphics[width=\textwidth]{sine1_1sec.png}
\caption{PQFM: Sine1}
\label{fig:sin1PQFM}
\end{subfigure}
\begin{subfigure}{0.5\textwidth}
\centering
\includegraphics[width=\textwidth]{sine2_2sec.png}
\caption{PQFM: Sine2}
\label{fig:sin2PQFM}
\end{subfigure}
\caption{Sine Waves}
\end{figure}


\begin{figure}[H]
\centering
\includegraphics[width=0.8\textwidth]{handel_5sec.png}
\caption{PQFM: Handel 5 seconds}
\label{fig:handPQFM}
\end{figure}

\begin{figure}[H]
\centering
\includegraphics[width=0.8\textwidth]{cast_5sec.png}
\caption{PQFM: Castanets 5 seconds}
\label{fig:castPQFM}
\end{figure}


\begin{figure}[H]
\centering
\includegraphics[width=0.8\textwidth]{gilberto_5sec.png}
\caption{PQFM: Gilberto 5 seconds}
\label{fig:gilbPQFM}
\end{figure}




The PQFM coefficients essentially show an outline of the frequency
spectrum of the various songs. 

As seen in Figure \ref{fig:samp1PQFM}, the extreme lower end of the PQFM
coefficients are low, while they get larger as you approach coefficient
number 5000. If one listens to the song, it can be easily heard how the
piano lacks a substantial bass, but the higher notes can be heard more
easily. 

Looking at Figure \ref{fig:samp2PQFM}, the lower end of the spectrum is
much more substantial, as can be heard by the thumping bass in the song.
Since the music is mostly electronic, it lacks many of the stronger
overtones that are commonly found on natural classical instruments. As you
go even higher in the spectrum, the coefficients do not approach zero as
quickly as was the case in Figure \ref{fig:samp1PQFM}. Perhaps this is due
to the inevitable higher-order harmonics that result from electronic
music. 

Figures \ref{fig:sin1PQFM} and \ref{fig:sin2PQFM} show the most
discernible distinctions. The first sine wave is obviously lower than the
second because of the larger concentration around the lower end of the
spectrum. 

Figure \ref{fig:handPQFM} is about the same as Figure \ref{fig:samp1PQFM}.
There
is more activity on the lower end of the spectrum though. This is probably
due to the fact that "handel.wav" includes vocals as well as piano and
violin. 


Figure \ref{fig:castPQFM} is quite interesting. While the previous figures had
a fairly consistent decline as the coefficients increased, this trend
seems fairly haphazard. Something of considerable note though is that
this is the first plot to have neighbouring coefficients of unequal magnitude.
While the other plots were extremely symmetrical about the 0 amplitude point, 
Figure \ref{fig:castPQFM} is not. Perhaps this is due to the highly percussive
nature of the castanets and its atonal sounds. 

In Figure \ref{fig:gilbPQFM}, you see a combination of Figures \ref{fig:handPQFM}
and \ref{fig:castPQFM}. In place of the castanets, we have the Brazilian 
tamborim. While it has a tonal center, when the tamborim is struck in rapid
succession it produces a much noisier sound that lacks a discernible tonal
center. The applause at the beginning of the song also adds some noisiness
that can be seen in Figure \ref{fig:gilbPQFM} by the large spikes at seemingly
random intervals. 


%%%%%%%%%%%%%%%%%%%%%%%%%%%%%%%%%%%%%%%%%%%%%%%%%%%%%%%


\section{Cosine Modulated Pseudo Quadrature Mirror Filter: Synthesis}

The synthesis, or reconstruction, from the coefficients is performed
in a very similar manner. The following equations yield the reconstruction
of 32 audio samples from 32 subband coefficients

\begin{align}
\label{eq:vshift}
\texttt{for } i=1023 \texttt{ down to } 64 \texttt{ do}\nonumber \\
v[i]=v[i-64]
\end{align}

\begin{align}
\label{eq:Nisk}
\texttt{for } i=63 \texttt{ down to } 0 \texttt{ do} \nonumber \\
v[i]=\sum\limits_{k=0}^{31} N_{i,k} \; s[k]
\end{align}

\begin{alignat}{2}
\texttt{for } i=&0 \texttt{ to } 7 \texttt{ do} \nonumber \\
\texttt{for } &j=0 \texttt{ to } 31 \texttt{ do} \nonumber \\
& u[64i+j] &=&v[128i+j] \label{eq:ushift1}\\
& u[64i+j+32] &= &v[128i+j+96]\label{eq:ushift2} 
\end{alignat}


\begin{align}
\texttt{for } i=0 \texttt{ to } 511 \texttt{ do} \nonumber \\
w[i]=d[i] u[i] \label{eq:wi}
\end{align}

\begin{align}
\texttt{for } j=0 \texttt{ to } 31 \texttt{ do} \nonumber \\
x[j]=\sum\limits_{j=0}^{15} w[j+32i] \label{eq:xj}
\end{align}

where

\begin{equation}
\label{eq:nik}
\boxed{
N_{i,k}=\cos \left( \frac{(2k+1)(16+i)\pi}{64}\right), 
\quad i=0,\dots,63, \; k=0,\dots,31
}
\end{equation}

%\lstinputlisting[caption=kullbackDistance.m,language=MATLAB,numbers=left,label=ls:KLdm]{./kullbackDistance.m} 

\begin{framed}
\textbf{Assignment}
\begin{enumerate}
\setcounter{enumi}{2}
\item Write the MATLAB function \verb|ipqmf| that implements the synthesis
filter bank described in equations \ref{eq:Nisk}-\ref{eq:nik}. 
The function will have the following template:
\\[2ex]
\verb|[recons] = ipqmf (coefficients)|
\\[2ex]

where \texttt{coefficients} is a buffer that contains the coefficients
computed by \texttt{pqmf}. The output array \texttt{recons} has the same
size as the buffer \texttt{coefficients}, and contains the reconstructed
audio data.

\item Reconstruct the first 5 seconds of the following tracks, and display
the signal \texttt{input}. You should observe that the reconstructed
signal is slightly delayed. This is due to the fact that the processing
assumes that a buffer of 512 audio samples is immediately available.

\begin{itemize}
\item sample1.wav, sample2.wav
\item sine1.wav, sine2.wav
\item handle.wav
\item gilberto.wav
\end{itemize}

\item Compute the maximum error between the reconstructed signal and the
original, taking into account the delay. The error should be no more than
$10^{-5}$. Explain how you estimate the delay. 

\end{enumerate}
\end{framed}

%%%%%%%%%%%%%%%%%%%%%%%%%%%%%%%%%%%%%%%% Figures Question 2

\begin{figure}[H]
\centering
\includegraphics[width=\textwidth]{sample1_ipqmf.png}
\caption{PQMF: Sample1 IPQMF}
\label{fig:samp1PQMF}
\end{figure}
\begin{figure}[H]
\centering
\includegraphics[width=\textwidth]{sample2_ipqmf.png}
\caption{PQMF: Sample2 IPQMF}
\label{fig:samp2PQMF}
\end{figure}


\begin{figure}[H]

\begin{subfigure}{0.5\textwidth}
\centering
\includegraphics[width=\textwidth]{sine1_ipqmf.png}
\caption{IPQMF: Sine1}
\label{fig:sin1IQMF}
\end{subfigure}
\begin{subfigure}{0.5\textwidth}
\centering
\includegraphics[width=\textwidth]{sine2_ipqmf.png}
\caption{IPQMF: Sine2}
\label{fig:sin2IQMF}
\end{subfigure}
\caption{Sine Waves}
\label{fig:sinIPQMF}
\end{figure}
 

\begin{figure}[H]
\centering
\includegraphics[width=0.8\textwidth]{handel_ipqmf.png}
\caption{IPQMF: Handel 5 seconds}
\label{fig:handIQMF}
\end{figure}

\begin{figure}[H]
\centering
\includegraphics[width=0.8\textwidth]{cast_ipqmf.png}
\caption{PQMF: Castanets 5 seconds}
\label{fig:castIQMF}
\end{figure}


\begin{figure}[H]
\centering
\includegraphics[width=0.8\textwidth]{gilberto_ipqmf.png}
\caption{PQMF: Gilberto 5 seconds}
\label{fig:gilbIQMF}
\end{figure}


As can be seen cleary in figures \ref{fig:sinIPQMF}-\ref{fig:gilbIQMF}, there are some obvious problems with the reconstruction, but
nonetheless the general form of the reconstruction follows the original audio. After a careful analysis of the issue, it could be seen that
there are very slight differences in the PQMF plot(s). While I originally tried a very systematic manner for computing the delay
between the original signal and the reconstruction, I eventually found it to be just as effective to view the delay visually from
a plot showing the original and the reconstruction. Reading through ``Pan95-mega.pdf'' reveals that 
\emph{The net offset is 320 points to time-align the psychoacoustic model data with the filter bank outputs}. This number
doesn't produce a good result though, so I stick with my experimental value instead. 


\lstinputlisting[caption=ipqmf.m,language=MATLAB,numbers=left,label=ls:ipqmf]
{./ipqmf.m} 


To compute the error between the original audio and the reconstruction, I simply added a slight bit on the end of my \verb|buildReport.m| file
as can be seen in Listing \ref{ls:totalError}.


\lstinputlisting[caption=buildReport.m,language=MATLAB,numbers=left,firstline=41,firstnumber=41,label=ls:totalError]{./buildReport.m}


%%%%%%%%%%%%%%%%%%%%%%%%%%%%%%%%% Question 6
\begin{framed}
\begin{enumerate}
\setcounter{enumi}{5}
\item Modify your code to reconstruct an audio signal using only a subset
of bands. This is the beginning of compression. Your function prototype
should look like this: 
\\[2ex]
\verb|[recons] = ipqmf (coefficients, thebands)|
\\[2ex]

where \verb|thebands| is an array of 32 integers, such that
\verb|thebands[i] =1| if band $i$ is used in the reconstruction, and
\verb|thebands[i] =0| if band $i$ is not used. 

Experiment with the files

\begin{itemize}
\item sample1.wav, sample2.wav
\item sine1.wav, sine2.wav
\item handel.wav
\item cast.wav
\item gilberto.wav
\end{itemize}

and describe the outcome of the experiments, when the certain bands are
not used to reconstruct. If you describe the compression ratio as

\begin{equation}
\frac{\texttt{number of bands used to reconstruct}}{32}
\end{equation}

explain what is a good compression ratio, and a good choice of bands for
each audio sample. Note that the psychoacoustic model of MP3 performs
this task automatically. 
\end{enumerate}
\end{framed}


After modifying the \verb|ipqmf| function, I used bands 1 through 7 and computed the reconstruction. 
Surprisingly, as can be seen in Figure \ref{fig:totErr}, the error is actually decreased. So far, this only
means that my reconstruction is so bad that no results are better than what I have. Further improvements
will lead to a more expected result. An expected result would be that the error increases as less bands are used. 
Right now, I have a compression rate of $6/32$ but that doesn't matter until my output is actually correct. 

The psychoacoustic model is designed so that frequencies that have lower bark values (similar to the mel values
computed in Lab 1) do not need to be included in the subband calculation. 


\begin{figure}[H]

\begin{subfigure}{0.5\textwidth}
\centering
\includegraphics[width=\textwidth]{totalError.png}
\caption{Total Error}
\label{fig:totErr0}
\end{subfigure}
\begin{subfigure}{0.5\textwidth}
\centering
\includegraphics[width=\textwidth]{totalError1.png}
\caption{Total Error (specialized subbands)}
\label{fig:totErr1}
\end{subfigure}
\caption{Total Errors}
\label{fig:totErr}
\end{figure}


%%%%%%%%%%%%%%%%%%%%%%%%%%%%%%%%%%%%%%%%%%%%%%%%%%%%%%%%%%%%%
\clearpage
\appendix
\section{Figures}

%%%%%%%%%%%%%%%%%%%%%%
\subsection{PQMF Models}
\begin{figure}[H]
\centering
\includegraphics[width=0.8\textwidth]{sample1_5sec.png}
\caption{PQFM: Sample1 5 seconds}
\end{figure}

\begin{figure}[H]
\centering
\includegraphics[width=0.8\textwidth]{sample2_5sec.png}
\caption{PQFM: Sample2 5 seconds}
\end{figure}


\begin{figure}[H]

\begin{subfigure}{0.5\textwidth}
\centering
\includegraphics[width=0.8\textwidth]{sine1_1sec.png}
\caption{PQFM: Sine1}
\end{subfigure}
\begin{subfigure}{0.5\textwidth}
\centering
\includegraphics[width=0.8\textwidth]{sine2_2sec.png}
\caption{PQFM: Sine2}
\end{subfigure}
\caption{Sine Waves}
\end{figure}


\begin{figure}[H]
\centering
\includegraphics[width=0.8\textwidth]{handel_5sec.png}
\caption{PQFM: Handel 5 seconds}
\end{figure}

\begin{figure}[H]
\centering
\includegraphics[width=0.8\textwidth]{cast_5sec.png}
\caption{PQFM: Castanets 5 seconds}
\end{figure}


\begin{figure}[H]
\centering
\includegraphics[width=0.8\textwidth]{gilberto_5sec.png}
\caption{PQFM: Gilberto 5 seconds}
\end{figure}
%%%%%%%%%%%%%%%%



%%%%%%%%%%%%%%%%%%%%%%%%%%%%%%%%%%%%%%%%%%%%%%%%%%%
\subsection{IPQMF Models}
\subsubsection{Normal}

\begin{figure}[H]
\centering
\includegraphics[width=0.8\textwidth]{sample1_ipqmf.png}
\caption{PQMF: Sample1 IPQMF}
\end{figure}
\begin{figure}[H]
\centering
\includegraphics[width=0.8\textwidth]{sample2_ipqmf.png}
\caption{PQMF: Sample2 IPQMF}
\end{figure}


\begin{figure}[H]

\begin{subfigure}{0.5\textwidth}
\centering
\includegraphics[width=\textwidth]{sine1_ipqmf.png}
\caption{IPQMF: Sine1}
\end{subfigure}
\begin{subfigure}{0.5\textwidth}
\centering
\includegraphics[width=\textwidth]{sine2_ipqmf.png}
\caption{IPQMF: Sine2}
\end{subfigure}
\caption{Sine Waves}
\end{figure}
 

\begin{figure}[H]
\centering
\includegraphics[width=0.8\textwidth]{handel_ipqmf.png}
\caption{IPQMF: Handel 5 seconds}
\end{figure}

\begin{figure}[H]
\centering
\includegraphics[width=0.8\textwidth]{cast_ipqmf.png}
\caption{PQMF: Castanets 5 seconds}
\end{figure}


\begin{figure}[H]
\centering
\includegraphics[width=0.8\textwidth]{gilberto_ipqmf.png}
\caption{PQMF: Gilberto 5 seconds}
\end{figure}

%%%%%%%%%%%%%%%%%%%%%%
\subsubsection{Delay Fixed}



\begin{figure}[H]
\centering
\includegraphics[width=0.8\textwidth]{sample1_D_ipqmf.png}
\caption{PQMF: Sample1 IPQMF}
\end{figure}
\begin{figure}[H]
\centering
\includegraphics[width=0.8\textwidth]{sample2_D_ipqmf.png}
\caption{PQMF: Sample2 IPQMF}
\end{figure}


\begin{figure}[H]

\begin{subfigure}{0.5\textwidth}
\centering
\includegraphics[width=\textwidth]{sine1_D_ipqmf.png}
\caption{IPQMF: Sine1}
\end{subfigure}
\begin{subfigure}{0.5\textwidth}
\centering
\includegraphics[width=\textwidth]{sine2_D_ipqmf.png}
\caption{IPQMF: Sine2}
\end{subfigure}
\caption{Sine Waves}
\end{figure}
 

\begin{figure}[H]
\centering
\includegraphics[width=0.8\textwidth]{handel_D_ipqmf.png}
\caption{IPQMF: Handel 5 seconds}
\end{figure}

\begin{figure}[H]
\centering
\includegraphics[width=0.8\textwidth]{cast_D_ipqmf.png}
\caption{PQMF: Castanets 5 seconds}
\end{figure}


\begin{figure}[H]
\centering
\includegraphics[width=0.8\textwidth]{gilberto_D_ipqmf.png}
\caption{PQMF: Gilberto 5 seconds}
\end{figure}

%%%%%%%%%%%%%%%%%%%%%%%%%%%%
\subsubsection{Delay Fixed Zoomed}

\begin{figure}[H]
\centering
\includegraphics[width=0.8\textwidth]{sample1_D1024_ipqmf.png}
\caption{PQMF: Sample1 IPQMF}
\end{figure}
\begin{figure}[H]
\centering
\includegraphics[width=0.8\textwidth]{sample2_D1024_ipqmf.png}
\caption{PQMF: Sample2 IPQMF}
\end{figure}


\begin{figure}[H]

\begin{subfigure}{0.5\textwidth}
\centering
\includegraphics[width=\textwidth]{sine1_D1024_ipqmf.png}
\caption{IPQMF: Sine1}
\end{subfigure}
\begin{subfigure}{0.5\textwidth}
\centering
\includegraphics[width=\textwidth]{sine2_D1024_ipqmf.png}
\caption{IPQMF: Sine2}
\end{subfigure}
\caption{Sine Waves}
\end{figure}
 

\begin{figure}[H]
\centering
\includegraphics[width=0.8\textwidth]{handel_D1024_ipqmf.png}
\caption{IPQMF: Handel 5 seconds}
\end{figure}

\begin{figure}[H]
\centering
\includegraphics[width=0.8\textwidth]{cast_D1024_ipqmf.png}
\caption{PQMF: Castanets 5 seconds}
\end{figure}


\begin{figure}[H]
\centering
\includegraphics[width=0.8\textwidth]{gilberto_D1024_ipqmf.png}
\caption{PQMF: Gilberto 5 seconds}
\end{figure}


%%%%%%%%%%%%%%%%%%%%%%%%%%%%%%%%
\subsubsection{Delay Fixed: Specialized Subband (Zoomed)}
\begin{figure}[H]
\centering
\includegraphics[width=0.8\textwidth]{sample1_DS1024_ipqmf.png}
\caption{PQMF: Sample1 IPQMF}
\end{figure}
\begin{figure}[H]
\centering
\includegraphics[width=0.8\textwidth]{sample2_DS1024_ipqmf.png}
\caption{PQMF: Sample2 IPQMF}
\end{figure}


\begin{figure}[H]

\begin{subfigure}{0.5\textwidth}
\centering
\includegraphics[width=\textwidth]{sine1_DS1024_ipqmf.png}
\caption{IPQMF: Sine1}
\end{subfigure}
\begin{subfigure}{0.5\textwidth}
\centering
\includegraphics[width=\textwidth]{sine2_DS1024_ipqmf.png}
\caption{IPQMF: Sine2}
\end{subfigure}
\caption{Sine Waves}
\end{figure}
 

\begin{figure}[H]
\centering
\includegraphics[width=0.8\textwidth]{handel_DS1024_ipqmf.png}
\caption{IPQMF: Handel 5 seconds}
\end{figure}

\begin{figure}[H]
\centering
\includegraphics[width=0.8\textwidth]{cast_DS1024_ipqmf.png}
\caption{PQMF: Castanets 5 seconds}
\end{figure}


\begin{figure}[H]
\centering
\includegraphics[width=0.8\textwidth]{gilberto_DS1024_ipqmf.png}
\caption{PQMF: Gilberto 5 seconds}
\end{figure}


%%%%%%%%%%%%%%%%%%%%%%%%%%%%%%%%%%
\section{Listings}

\lstinputlisting[caption=pqmf.m,language=MATLAB,numbers=left]
{./pqmf.m} 

\lstinputlisting[caption=ipqmf.m,language=MATLAB,numbers=left]
{./ipqmf.m} 

\lstinputlisting[caption=buildReport.m,language=MATLAB,numbers=left]
{./buildReport.m} 



\end{document}