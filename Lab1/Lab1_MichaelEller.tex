% !TEX TS-program = pdflatex
% !TEX encoding = UTF-8 Unicode

% This is a simple template for a LaTeX document using the "article" class.
% See "book", "report", "letter" for other types of document.

\documentclass{article} % use larger type; default would be 10pt

\usepackage[utf8]{inputenc} % set input encoding (not needed with XeLaTeX)

%%% Examples of Article customizations
% These packages are optional, depending whether you want the features they provide.
% See the LaTeX Companion or other references for full information.

%%% PAGE DIMENSIONS
\usepackage{geometry} % to change the page dimensions
\geometry{letterpaper} % or letterpaper (US) or a5paper or....
 \geometry{margin=1in} % for example, change the margins to 2 inches all round
% \geometry{landscape} % set up the page for landscape
%   read geometry.pdf for detailed page layout information

\usepackage{graphicx} % support the \includegraphics command and options
\usepackage[normalem]{ulem}
% \usepackage[parfill]{parskip} % Activate to begin paragraphs with an empty line rather than an indent
%\usepackage[subsection]{placeins}
\usepackage{float}
\usepackage [autostyle, english = american]{csquotes}
%%% PACKAGES
\usepackage{booktabs} % for much better looking tables
\usepackage{array} % for better arrays (eg matrices) in maths
\usepackage{paralist} % very flexible & customisable lists (eg. enumerate/itemize, etc.)
\usepackage{verbatim} % adds environment for commenting out blocks of text & for better verbatim
\usepackage{subfig} % make it possible to include more than one captioned figure/table in a single float
% These packages are all incorporated in the memoir class to one degree or another...
\usepackage{mcode}
%%% HEADERS & FOOTERS
\usepackage{fancyhdr} % This should be set AFTER setting up the page geometry
\pagestyle{fancy} % options: empty , plain , fancy
\renewcommand{\headrulewidth}{0pt} % customise the layout...
\lhead{}\chead{}\rhead{}
\lfoot{}\cfoot{\thepage}\rfoot{}

%%% SECTION TITLE APPEARANCE
\usepackage{sectsty}
\allsectionsfont{\sffamily\mdseries\upshape} % (See the fntguide.pdf for font help)
% (This matches ConTeXt defaults)

%%% ToC (table of contents) APPEARANCE
\usepackage[nottoc,notlof,notlot]{tocbibind} % Put the bibliography in the ToC
\usepackage[titles,subfigure]{tocloft} % Alter the style of the Table of Contents
\renewcommand{\cftsecfont}{\rmfamily\mdseries\upshape}
\renewcommand{\cftsecpagefont}{\rmfamily\mdseries\upshape} % No bold!
\renewcommand{\topfraction}{0.85}
\renewcommand{\textfraction}{0.1}
\renewcommand{\floatpagefraction}{0.85}

%%% END Article customizations
\usepackage{float}
\usepackage{framed}
\usepackage{amsfonts}
\usepackage{amsmath}
\usepackage{mathabx}
%%% The "real" document content comes below...
\newcommand{\includecode}[2][c]{\lstinputlisting[caption=#2, escapechar=, style=custom#1]{#2}<!---->}
\title{Lab 1}
\author{Michael Eller}
\date{February 1, 2016} % Activate to display a given date or no date (if empty),
         	      % otherwise the current date is printed 

\begin{document}
\maketitle

\tableofcontents
\clearpage
\section{Introduction}
The goal of this lab is to explore different methods of categorizing 
music into specific genres. Several methods of automated music 
analysis exist already, including \textbf{score following}, 
\textbf{automatic music transcription}, 
\textbf{music recommendation}, and \textbf{machine listening}. 
Unfortunately, all of these tools are still rudimentary. 
Faster, more efficient, methods still need to be developed in order
to allow automated music analysis to be implemented on portable music
players. The goal of this lab is to develop tools to automatically
extract some defining features of music that will help us to
categorize them more easily. 
\section{Sampling Rates}
Current high-quality audio standards have a 24-bit depth and is sampled at 96 kHz. This 24-bit depth means there are 16,777,216 possible
values for the audio signal at any given instance. It also means we are able to replicate frequencies up to 48 kHz. While this is far
above the 20 kHz limit of human hearing, DVD audio is simply not high enough quality enough for a dolphin to listen to. 

\begin{framed}
\textbf{Assignment}
\begin{enumerate}
\item Dolphins can only hear sounds over the frequency range [7 - 120] kHz. At what sampling
frequency $f_s$ should we sample digital audio signals for dolphins?
\end{enumerate}
\end{framed}

Nyquist theorem states that in order to accurately recreate a signal with maximum frequency $f_s/2$, we must sample at a minimum frequency of $f_s$. 
Therefore, our sampling frequency for digital audio signals for dolphins should be at minimum 240 kHz. Currently, one of the highest portable audio
standards of BD-ROM LPCM (lossless) allow for 24 bit/sample and a maximum sampling frequency of only 192 kHz. Less common standards do
exist though: Digital eXtreme Definition at 352.8 kHz using for recording and editing Super Audio CDs, SACD at 2,822.4 kHz known as Direct Stream 
Digital, and Double-Rate DSD at 5,644.8 kHz used in some professional DSD recorders. 
Therefore, as it stands, we might want to stay away from producing audio for dolphins (at least until better recording standards are developed. 

\section{Audio signal}

\begin{framed}
\textbf{Assignment}
\begin{enumerate}
\setcounter{enumi}{1}
\item Write a MATLAB function that extract T seconds of music from a given track. You
will use the MATLAB function \sout{waveread} audioread to read a track and the function play to
listen to the track.\\\\
In the lab you will use T =\textbf{24 seconds} from the middle of each track to compare the
different algorithms. Download the files, and test your function.
\end{enumerate}
\end{framed}

\lstinputlisting[caption=extractSound.m,language=MATLAB,numbers=left]{./extractSound.m}

This MATLAB function is fairly straightforward. MATLAB's built-in function \emph{audioinfo} provided all the necessary attributes to allow my function to parse
any .wav file and extract the number of samples needed. 

\section{Low Level Features: Time-Domain Analysis}
The bulk of music analysis is done in the frequency spectrum, however, 
there are some low level features that can be found from the time-domain
analysis of music as well. 

\begin{framed}
\textbf{Assignment}
\begin{enumerate}
\setcounter{enumi}{2}
\item Implement the loudness and ZCR and evaluate these features on the different music
tracks. Your MATLAB function should display each feature as a time series in a separate
figure.
\item Comment on the specificity of the feature and its ability to separate different musical genre. 
\end{enumerate}
\end{framed}

\subsection{Loudness}
There is not an easy way to describe \emph{loudness}, but 
one way to mathematically quantize it is by finding it
standard deviation, $\sigma(n)$. \\
\begin{equation}\label{eq:loud_std}
\sigma(n)=\sqrt{\frac{1}{N-1} \sum\limits_{m=-N/2}^{N/2-1} 
[ x(n+m)-\mathbb{E}[x_n]]^2 }\hspace{5pt} \text{ with }\hspace{5pt} 
\mathbb{E}[x_n]=\frac{1}{N}\sum\limits_{m=-N/2}^{N/2-1}x(n+m)
\end{equation}


The following is an excerpt from my \emph{loudness} function. My \emph{extractSound} function extracts 24 seconds around the middle of the song. 
The song excerpt is then split into frames of size 255 overlapped (at least) halfway with the previous frame. 

\lstinputlisting[caption=loudness.m,language=MATLAB,numbers=left,firstline=30,lastline=32,firstnumber=30]{./loudness.m}


\begin{figure}[h]
\centering
\includegraphics[width=\textwidth]{Loudness_track201-classical.png}
\caption{Loudness value per frame}
\label{fig:loudness}
\end{figure}

Figure \ref{fig:loudness} shows an example of the function applied to \lq\lq{}track201-classical.wav\rq\rq{}. The figure would imply that, for the selection  of the song, the beginning is quite 
loud, then quiets down until two loud moments, and finishes at a moderately quiet section. If one listens to the song selection in question, the results of my loudness function can be
confirmed. The other loudness plots can be found in Appendix \ref{sec:LoudnessFigures}

The \textbf{loudness} of a track appears to be most helpful in identifying \emph{classical} and \emph{jazz} music. 
Classical is very fluid and will not transition quickly between a high vs a low loudness. 
Jazz is a very periodic music and that is evident by the loudness plots where there are periodic spikes with a lull in between. 


\subsection{Zero-Crossing Rate (ZCR)}

Another low-level feature that can be gathered from a time-domain analysis is the Zero-Cross Rate (ZCR). This feature has been used heavily in both speech recognition and music 
information retrieval, being a key feature to classify percussive sounds, where a frequency analysis would not be able to identify such features. 

\begin{equation}\label{eq:ZCR}
ZCR(n)=\frac{1}{2N}\sum\limits_{m=-N/2}^{N/2-1}|sgn(x(n+m))-sgn(x(n+m-1))|
\end{equation}

The following is an excerpt from my \emph{zeroCross} function. Once again, I extract the middle 24 seconds of audio from my track and store it into frames of size 255. 
Once \textbf{ZCR\_data} has been allocated, each adjacent sample is compared to each other to check if they have crossed the zero-axis. 

\lstinputlisting[caption=zeroCross.m,language=MATLAB,numbers=left,firstline=10,lastline=13,firstnumber=10]{./zeroCross.m}


\begin{figure}[H]
\centering
\includegraphics[width=\textwidth]{ZCR_track201-classical.png}
\caption{Average ZCR per frame}
\label{fig:ZCR}
\end{figure}


Figure \ref{fig:ZCR} shows an example of the ZCR function applied to \lq\lq{}track201-classical.wav\rq\rq{}. The plot produced by the ZCR function appears similar in shape to the plot from
the previous loudness function. The ZCR function is more closely representative of how \textbf{noisy} a function is though. 
The other ZCR plots can be found in Appendix \ref{sec:ZCRFigures}. 

Just as the \textbf{loudness} function, the \textbf{ZCR} function appears to be most helpful in identifying \emph{classical} and \emph{jazz} music. 
There doesn't seem to be any advantage in using \textbf{loudness} vs \textbf{ZCR} in identifying music genres. One method may prove to be faster on a large-scale integration,
but such results cannot be determined on such a small unit test such as this one. 

\section{Low Level Features: Spectral Analysis}
In this section, our goal is to reconstruct a musical score based on the spectral analysis of a piece. If we can apply the proper window function to each frame, we might be able
to ascertain which note(s) are being played at that specific moment. 

To properly evaluate the frequency domain of a frame, we need to perform the following operations:

\begin{framed}
\begin{equation}
\label{eq:window}
\begin{aligned}
Y=\text{FFT}(w.*x_n);\\
K=N/2+1;\\
X_n=Y(1:K);\\
\end{aligned}
\end{equation}
\end{framed}
where $w$ is our windowing function and $N$ is the number of samples per frame.

\begin{framed}
\textbf{Assignment}
\begin{enumerate}
\setcounter{enumi}{4}
\item Let
\begin{equation}
\label{eq:cosFT}
x[n]=\cos(\omega_0 n), n \in\mathbb{Z}
\end{equation}
Derive the theoretical expression of the discrete time Fourier transform of $x$, given by
\begin{equation}
\label{eq:DFT}
X(e^{j \omega})=\sum\limits_{n=-\infty}^{\infty} x[n]e^{-j \omega n}
\end{equation}
\end{enumerate}
\end{framed}

The Fourier transform of a cosine function is trivial at this point. A cosine function can be regarded as a sum of two complex exponentials. And the Fourier transform of a complex
exponential is simply the dirac delta function offset by frequency $\omega$. 
Therefore, the Fourier transform of $x[n]=\cos(\omega_0 n), n \in\mathbb{Z}$ is:\\
\[ \sqrt{\frac{\pi}{2}}\delta(\omega-\omega_0)+\sqrt{\frac{\pi}{2}}\delta(\omega+\omega_0)\]


\begin{framed}
\textbf{Assignment}
\begin{enumerate}
\setcounter{enumi}{5}
\item In practice, we work with a finite signal, and we multiply the signal $x[n]$ by a window $w[n]$. We assume that the window $w$ is non zero at times $n=-N/2,\dots,N/2$, 
and we define
\begin{equation} y[n]=x[n]w[n-N/2] \end{equation}
Derive the expression of the Fourier transform $y[n]$ in terms of the Fourier transform of $x$ and the Fourier transform of the window $w$. 
\end{enumerate}
\end{framed}

From prior knowledge, we know that the Fourier transform multiplication of two functions in time domain is equal to the periodic convolution of two functions in the Fourier domain, as follows:
\[\text{if } y[n]=x[n]\cdot h[n] \]
\[\text{ then, }\\
Y(e^{j\omega})=\frac{1}{2\pi}\int_{-\pi}^{\pi}X(e^{j\theta}) \cdot H(e^{j(\omega-\theta)}) d\theta
\]
We also know that a shift in time equates to a scaling in the frequency domain:
\[\mathcal{F}(x[n-k])=X(e^{j\omega})e^{-j\omega k}
\]
From these two properties, we can equate the DTFT of $y[n]$. 

\[\text{if } y[n]=x[n]w[n-N],\]
\[\text{then } Y(e^{j\omega})=\frac{1}{2\pi}\int_{-\pi}^{\pi}X(e^{j\theta}) \cdot (H(e^{j(\omega-\theta)})\cdot e^{-j\omega N}) d\theta\]



\begin{framed}
\textbf{Assignment}
\begin{enumerate}
\setcounter{enumi}{6}
\item Implement the computation of the windowed Fourier transform of $y$, given by Equation \ref{eq:window}.
Evaluate its performance with pure sinusoidal signals and different windows:
\begin{itemize}
\item Bartlett
\item Hann
\item Kaiser
\end{itemize}
\item Compute the spectrogram of an audio track as follows:
\begin{enumerate}
\item Decompose a track into a sequence of $N_f$ overlapping \emph{frames} of size $N$. The overlap between two frames should be $N/2$. 
\item Compute the magnitude squared of the Fourier transform, $|X(k)|^2,k=1,\dots,K$ over each frame $n$. 
\item Display the Fourier transform of all the frames in a matrix of size $K \times N_f$.
\end{enumerate}
You will experiment with different audio tracks, as well as pure sinusoidal tones. Do the spectrograms like what you hear?
\end{enumerate}
In the rest of the lab, we will be using a Kaiser window to compute the Fourier transform, as explained in Equation \ref{eq:window}.
\end{framed}

In order to nicely bridge the gap between our continuous time perception of sound and a computer's discrete time processing, a windowing function is often used. 
This way, we are able to recreate the desired signal as close as possible without generating any high frequency components that result from abruptly stopping a time sample. 
In Figure \ref{fig:windowsTiming} below, we are able to see the performance differences between each window. Overall, the Bartmann and Hamming windows seemed to perform
the fastest, albeit by only a few milliseconds. 

\begin{figure}[H]
\centering
\includegraphics[width=0.75\textwidth]{windowsTiming.png}
\caption{Timing comparison of different windows}
\label{fig:windowsTiming}
\end{figure}

\lstinputlisting[caption=windows.m,language=MATLAB,numbers=left,firstline=15,firstnumber=15]{./windows.m}

In addition to how quickly each window is able to operate, we can also see how effectively each window extracts our signal. This can be seen in figures \ref{fig:bartWindow},
\ref{fig:hamWindow}, and \ref{fig:kaiserWindow}.


\begin{figure}[H]
\centering
\includegraphics[width=0.75\textwidth]{BartlettWindow.png}
\caption{Spectrogram of Bartlett Window}
\label{fig:bartWindow}
\end{figure}


\begin{figure}[H]
\centering
\includegraphics[width=0.75\textwidth]{HammingWindow.png}
\caption{Spectrogram of Hamming Window}
\label{fig:hamWindow}
\end{figure}

\begin{figure}[H]
\centering
\includegraphics[width=0.75\textwidth]{KaiserWindow.png}
\caption{Spectrogram of Kaiser Window}
\label{fig:kaiserWindow}
\end{figure}

In addition to being fairly fast, the Bartlett window seemed to be the most effective window in that its power to frequency ratio was very low away from the target frequency of 440 Hz. 


\begin{framed}
\textbf{Assignment}
\begin{enumerate}
\setcounter{enumi}{8}
\item Implement all the low level spectral features them on the different tracks. Your MATLAB function should display each feature as a time series in separate figure. 
\item Comment on the specificity of the feature, and its ability to separate different musical genres. 
\end{enumerate}
\end{framed}

In addition to a song having low-level time-domain characteristics, low-level frequency-domain characteristics exist as well. With modern DSP chips, computing an FFT is no longer
that big of an issue for DSP engineers. More plot for frequency centroids and frequency spreads can be found in Appendices \ref{sec:centroid} and \ref{sec:spread}, respectively.

\begin{figure}[H]
\centering
\includegraphics[width=\textwidth]{freqCenttrack201-classical.png}
\caption{Frequency Centroid}
\label{fig:cent201}
\end{figure}

\begin{figure}
\centering
\includegraphics[width=\textwidth]{freqSpreadtrack201-classical.png}
\caption{Frequency Spread}
\label{fig:spread_201}
\end{figure}



Just as the in the time-domain analyses, the frequency-domain functions appear to be most helpful in identifying \emph{classical} and \emph{jazz} music. 
There doesn't seem to be any advantage in using \textbf{centroid} vs \textbf{spread} in identifying music genres. One method may eventually be better, but it is unclear at the moment. 

Also, despite a more in depth analysis, these lower-level spectrum analysis tools seem to provide the same (if not less) help than the simple time-domain analyses. 

\begin{figure}[H]
\centering
\includegraphics[width=\textwidth]{specFluxtrack201-classical.png}
\caption{Spectral Flux}
\label{fig:flux_201}
\end{figure}

\begin{figure}[H]
\centering
\includegraphics[width=\textwidth]{specFlat_track201-classical.png}
\caption{Spectral Flatness}
\label{fig:flat_201}
\end{figure}

\section{Basic Psychoacoustic Quantities}
Humans are very good at identifying music genres. Our problem is that we are too slow at it though. Humans generally identify a specific genre of music with its subjective features
such as timbre, melody, harmony, rhythm, tempo, mood, lyrics, etc. In this section, we will focus solely on the aspects that are mathematically quantifiable. 

\begin{framed}
\textbf{Assignment}
\begin{enumerate}
\setcounter{enumi}{10}
\item Implement the computation of the triangular filterbanks $H_p,p=1,\dots ,N_B$. Your function will return an array $\mathbf{fbank}$ of size $N_B \times K$ such that
$\mathbf{fbank(p,:)}$ contains the filter bank $H_p$. 
\item Implement the computation of the mfcc coefficients, as defined by:
\begin{equation}
\text{mfcc}[p]=\sum\limits_{k=1}^{K}|H_p(k)X_n(k)|^2
\end{equation}
\end{enumerate}
\end{framed}

With a proper filter bank created, we are one step closer to categorizing music into genres. Now that we are using a Mel scale, hopefully we will be able to categorize the pitches
more closely to how a human ear would hear them. As opposed to a computer simply processing raw data. 

\lstinputlisting[caption=melBank.m,language=MATLAB,numbers=left,lastline=36]{./melBank.m}
In addition to having a working filter bank, we also have our Mel-coefficients. These may prove useful in future labs this semester. 
\lstinputlisting[caption=mfcc.m,language=MATLAB,numbers=left]{./mfcc.m}

\begin{figure}[H]
\centering
\includegraphics[width=\textwidth]{melFilterBank.png}
\caption{Mel Filter Bank}
\label{fig:mel_Bank}
\end{figure}







\clearpage
\appendix
\section{Figures}
\subsection{Loudness}
\label{sec:LoudnessFigures}

\begin{figure}[h]
\centering
\includegraphics[width=0.75\textwidth]{Loudness_track201-classical.png}
\caption{Loudness value per frame, classical201}
\label{fig:loudness_201}
\end{figure}

%\setcounter{figure}{2}
\begin{figure}
\centering
\vspace{-400em}
\includegraphics[width=0.75\textwidth]{Loudness_track204-classical.png}
\caption{Loudness value per frame, classical204}
\label{fig:loudness_204}
\end{figure}

\begin{figure}
\centering
\includegraphics[width=0.75\textwidth]{Loudness_track370-electronic.png}
\caption{Loudness value per frame, electronic370}
\label{fig:loudness_370}
\end{figure}

\begin{figure}
\centering
\includegraphics[width=0.75\textwidth]{Loudness_track396-electronic.png}
\caption{Loudness value per frame, electronic396}
\label{fig:loudness_396}
\end{figure}

\begin{figure}
\centering
\includegraphics[width=0.75\textwidth]{Loudness_track437-jazz.png}
\caption{Loudness value per frame, jazz437}
\label{fig:loudness_437}
\end{figure}

\begin{figure}
\centering
\includegraphics[width=0.75\textwidth]{Loudness_track439-jazz.png}
\caption{Loudness value per frame, jazz439}
\label{fig:loudness_439}
\end{figure}

\begin{figure}
\centering
\includegraphics[width=0.75\textwidth]{Loudness_track463-metal.png}
\caption{Loudness value per frame, metal463}
\label{fig:loudness_463}
\end{figure}


\begin{figure}
\centering
\includegraphics[width=0.75\textwidth]{Loudness_track492-metal.png}
\caption{Loudness value per frame, metal492}
\label{fig:loudness_492}
\end{figure}


\begin{figure}
\centering
\includegraphics[width=0.75\textwidth]{Loudness_track547-rock.png}
\caption{Loudness value per frame, rock547}
\label{fig:loudness_547}
\end{figure}


\begin{figure}
\centering
\includegraphics[width=0.75\textwidth]{Loudness_track550-rock.png}
\caption{Loudness value per frame, rock550}
\label{fig:loudness_550}
\end{figure}


\begin{figure}
\centering
\includegraphics[width=0.75\textwidth]{Loudness_track707-world.png}
\caption{Loudness value per frame, world707}
\label{fig:loudness_707}
\end{figure}


\begin{figure}[h!]
\centering
\includegraphics[width=0.75\textwidth]{Loudness_track729-world.png}
\caption{Loudness value per frame, world729}
\label{fig:loudness_729}
\end{figure}

\clearpage
\subsection{Zero-Cross Rate}
\label{sec:ZCRFigures}

%\setcounter{figure}{1}
\begin{figure}[ht!]
%\vspace{-300em}
\centering
\includegraphics[width=0.75\textwidth]{ZCR_track201-classical.png}
\caption{ZCR value per frame, classical201}
\label{fig:ZCR_201}
\end{figure}

%\setcounter{figure}{3}
\begin{figure}[!hb]
\centering
\includegraphics[width=0.75\textwidth]{ZCR_track204-classical.png}
\caption{ZCR value per frame, classical204}
\label{fig:ZCR_204}
\end{figure}

\begin{figure}
\centering
\includegraphics[width=0.75\textwidth]{ZCR_track370-electronic.png}
\caption{ZCR value per frame, electronic370}
\label{fig:ZCR_370}
\end{figure}

\begin{figure}
\centering
\includegraphics[width=0.75\textwidth]{ZCR_track396-electronic.png}
\caption{ZCR value per frame, electronic396}
\label{fig:ZCR_396}
\end{figure}

\begin{figure}
\centering
\includegraphics[width=0.75\textwidth]{ZCR_track437-jazz.png}
\caption{ZCR value per frame, jazz437}
\label{fig:ZCR_437}
\end{figure}

\begin{figure}
\centering
\includegraphics[width=0.75\textwidth]{ZCR_track439-jazz.png}
\caption{ZCR value per frame, jazz439}
\label{fig:ZCR_439}
\end{figure}

\begin{figure}
\centering
\includegraphics[width=0.75\textwidth]{ZCR_track463-metal.png}
\caption{ZCR value per frame, metal463}
\label{fig:ZCR_463}
\end{figure}


\begin{figure}
\centering
\includegraphics[width=0.75\textwidth]{ZCR_track492-metal.png}
\caption{ZCR value per frame, metal492}
\label{fig:ZCR_492}
\end{figure}


\begin{figure}
\centering
\includegraphics[width=0.75\textwidth]{ZCR_track547-rock.png}
\caption{ZCR value per frame, rock547}
\label{fig:ZCR_547}
\end{figure}


\begin{figure}
\centering
\includegraphics[width=0.75\textwidth]{ZCR_track550-rock.png}
\caption{ZCR value per frame, rock550}
\label{fig:ZCR_550}
\end{figure}


\begin{figure}
\centering
\includegraphics[width=0.75\textwidth]{ZCR_track707-world.png}
\caption{ZCR value per frame, world707}
\label{fig:ZCR_707}
\end{figure}


\begin{figure}[h!]
\centering
\includegraphics[width=0.75\textwidth]{ZCR_track729-world.png}
\caption{ZCR value per frame, world729}
\label{fig:ZCR_729}
\end{figure}

\clearpage
\subsection{Windows}

\begin{figure}[ht!]
\centering
\includegraphics[width=0.75\textwidth]{windowsTiming.png}
\caption{Timing comparison of different windows}
\label{fig:windowsTimings}
\end{figure}


\begin{figure}[H]
\centering
\includegraphics[width=0.75\textwidth]{BartlettWindow.png}
\caption{Spectrogram of Bartlett Window}
\label{fig:bartWindows}
\end{figure}


\begin{figure}[H]
\centering
\includegraphics[width=0.75\textwidth]{HammingWindow.png}
\caption{Spectrogram of Hamming Window}
\label{fig:hamWindows}
\end{figure}

\begin{figure}[H]
\centering
\includegraphics[width=0.75\textwidth]{KaiserWindow.png}
\caption{Spectrogram of Kaiser Window}
\label{fig:kaiserWindows}
\end{figure}

\clearpage
\subsection{Centroid}
\label{sec:centroid};

\begin{figure}[H]
\centering
\includegraphics[width=0.75\textwidth]{freqCenttrack201-classical.png}
\caption{Frequency Centroid}
\label{fig:centr201}
\end{figure}

\begin{figure}[H]
\centering
\includegraphics[width=0.75\textwidth]{freqCenttrack204-classical.png}
\caption{Frequency Centroid}
\label{fig:centr204}
\end{figure}

\begin{figure}[H]
\centering
\includegraphics[width=0.75\textwidth]{freqCenttrack370-electronic.png}
\caption{Frequency Centroid}
\label{fig:cent370}
\end{figure}


\begin{figure}[H]
\centering
\includegraphics[width=0.75\textwidth]{freqCenttrack396-electronic.png}
\caption{Frequency Centroid}
\label{fig:cent396}
\end{figure}


\begin{figure}[H]
\centering
\includegraphics[width=0.75\textwidth]{freqCenttrack463-metal.png}
\caption{Frequency Centroid}
\label{fig:cent463}
\end{figure}


\begin{figure}[H]
\centering
\includegraphics[width=0.75\textwidth]{freqCenttrack492-metal.png}
\caption{Frequency Centroid}
\label{fig:cent492}
\end{figure}


\begin{figure}[H]
\centering
\includegraphics[width=0.75\textwidth]{freqCenttrack547-rock.png}
\caption{Frequency Centroid}
\label{fig:cent547}
\end{figure}


\begin{figure}[H]
\centering
\includegraphics[width=0.75\textwidth]{freqCenttrack550-rock.png}
\caption{Frequency Centroid}
\label{fig:cent550}
\end{figure}


\begin{figure}[H]
\centering
\includegraphics[width=0.75\textwidth]{freqCenttrack729-world.png}
\caption{Frequency Centroid}
\label{fig:cent729}
\end{figure}


\begin{figure}[H]
\centering
\includegraphics[width=0.75\textwidth]{freqCenttrack707-world.png}
\caption{Frequency Centroid}
\label{fig:cent707}
\end{figure}



\clearpage
\subsection{Spread}
\label{sec:spread}

\begin{figure}[H]
\centering
\includegraphics[width=0.75\textwidth]{freqSpreadtrack201-classical.png}
\caption{Frequency Spread}
\label{fig:spread201}
\end{figure}

\begin{figure}[H]
\centering
\includegraphics[width=0.75\textwidth]{freqSpreadtrack204-classical.png}
\caption{Frequency Spread}
\label{fig:spread204}
\end{figure}

\begin{figure}[H]
\centering
\includegraphics[width=0.75\textwidth]{freqSpreadtrack370-electronic.png}
\caption{Frequency Spread}
\label{fig:spread370}
\end{figure}


\begin{figure}[H]
\centering
\includegraphics[width=0.75\textwidth]{freqSpreadtrack396-electronic.png}
\caption{Frequency Spread}
\label{fig:spread396}
\end{figure}


\begin{figure}[H]
\centering
\includegraphics[width=0.75\textwidth]{freqSpreadtrack463-metal.png}
\caption{Frequency Spread}
\label{fig:spread463}
\end{figure}


\begin{figure}[H]
\centering
\includegraphics[width=0.75\textwidth]{freqSpreadtrack492-metal.png}
\caption{Frequency Spread}
\label{fig:spread492}
\end{figure}


\begin{figure}[H]
\centering
\includegraphics[width=0.75\textwidth]{freqSpreadtrack547-rock.png}
\caption{Frequency Spread}
\label{fig:spread547}
\end{figure}


\begin{figure}[H]
\centering
\includegraphics[width=0.75\textwidth]{freqSpreadtrack550-rock.png}
\caption{Frequency Spread}
\label{fig:spread550}
\end{figure}


\begin{figure}[H]
\centering
\includegraphics[width=0.75\textwidth]{freqSpreadtrack729-world.png}
\caption{Frequency Spread}
\label{fig:spread729}
\end{figure}


\begin{figure}[H]
\centering
\includegraphics[width=0.75\textwidth]{freqSpreadtrack707-world.png}
\caption{Frequency Spread}
\label{fig:spread707}
\end{figure}


\subsection{Flatness}

\begin{figure}[H]
\centering
\includegraphics[width=0.75\textwidth]{specFlat_track201-classical.png}
\caption{Spectral Flatness}
\label{fig:flat201}
\end{figure}

\begin{figure}[H]
\centering
\includegraphics[width=0.75\textwidth]{specFlat_track204-classical.png}
\caption{Spectral Flatness}
\label{fig:flat204}
\end{figure}

\begin{figure}[H]
\centering
\includegraphics[width=0.75\textwidth]{specFlat_track370-electronic.png}
\caption{Spectral Flatness}
\label{fig:flat370}
\end{figure}


\begin{figure}[H]
\centering
\includegraphics[width=0.75\textwidth]{specFlat_track396-electronic.png}
\caption{Spectral Flatness}
\label{fig:flat396}
\end{figure}


\begin{figure}[H]
\centering
\includegraphics[width=0.75\textwidth]{specFlat_track463-metal.png}
\caption{Spectral Flatness}
\label{fig:flat463}
\end{figure}


\begin{figure}[H]
\centering
\includegraphics[width=0.75\textwidth]{specFlat_track492-metal.png}
\caption{Spectral Flatness}
\label{fig:flat492}
\end{figure}


\begin{figure}[H]
\centering
\includegraphics[width=0.75\textwidth]{specFlat_track547-rock.png}
\caption{Spectral Flatness}
\label{fig:flat547}
\end{figure}


\begin{figure}[H]
\centering
\includegraphics[width=0.75\textwidth]{specFlat_track550-rock.png}
\caption{Spectral Flatness}
\label{fig:flat550}
\end{figure}


\begin{figure}[H]
\centering
\includegraphics[width=0.75\textwidth]{specFlat_track729-world.png}
\caption{Spectral Flatness}
\label{fig:flat729}
\end{figure}


\begin{figure}[H]
\centering
\includegraphics[width=0.75\textwidth]{specFlat_track707-world.png}
\caption{Spectral Flatness}
\label{fig:flat707}
\end{figure}



\subsection{Flux}

\begin{figure}[H]
\centering
\includegraphics[width=0.75\textwidth]{specFluxtrack201-classical.png}
\caption{Spectral Flux}
\label{fig:flux201}
\end{figure}

\begin{figure}[H]
\centering
\includegraphics[width=0.75\textwidth]{specFluxtrack204-classical.png}
\caption{Spectral Flux}
\label{fig:flux204}
\end{figure}

\begin{figure}[H]
\centering
\includegraphics[width=0.75\textwidth]{specFluxtrack370-electronic.png}
\caption{Spectral Flux}
\label{fig:flux370}
\end{figure}


\begin{figure}[H]
\centering
\includegraphics[width=0.75\textwidth]{specFluxtrack396-electronic.png}
\caption{Spectral Flux}
\label{fig:flux396}
\end{figure}


\begin{figure}[H]
\centering
\includegraphics[width=0.75\textwidth]{specFluxtrack463-metal.png}
\caption{Spectral Flux}
\label{fig:flux463}
\end{figure}


\begin{figure}[H]
\centering
\includegraphics[width=0.75\textwidth]{specFluxtrack492-metal.png}
\caption{Spectral Flux}
\label{fig:flux492}
\end{figure}


\begin{figure}[H]
\centering
\includegraphics[width=0.75\textwidth]{specFluxtrack547-rock.png}
\caption{Spectral Flux}
\label{fig:flux547}
\end{figure}


\begin{figure}[H]
\centering
\includegraphics[width=0.75\textwidth]{specFluxtrack550-rock.png}
\caption{Spectral Flux}
\label{fig:flux550}
\end{figure}


\begin{figure}[H]
\centering
\includegraphics[width=0.75\textwidth]{specFluxtrack729-world.png}
\caption{Spectral Flux}
\label{fig:flux729}
\end{figure}


\begin{figure}[H]
\centering
\includegraphics[width=0.75\textwidth]{specFluxtrack707-world.png}
\caption{Spectral Flux}
\label{fig:flux707}
\end{figure}

\begin{figure}[H]
\centering
\includegraphics[width=0.75\textwidth]{melFilterBank.png}
\caption{Mel Filter Bank}
\label{fig:melBank}
\end{figure}

\clearpage
\section{Code}

\lstinputlisting[caption=extractSound.m,language=MATLAB,numbers=left]{./extractSound.m}

\lstinputlisting[caption=loudness.m,language=MATLAB,numbers=left]{./loudness.m}
\lstinputlisting[caption=loudnessPlot.m,language=MATLAB,numbers=left]{./loudnessPlot.m}

\lstinputlisting[caption=zeroCross.m,language=MATLAB,numbers=left]{./zeroCross.m}
\lstinputlisting[caption=zeroPlot.m,language=MATLAB,numbers=left]{./zeroPlot.m}

\lstinputlisting[caption=windows.m,language=MATLAB,numbers=left]{./windows.m}
\lstinputlisting[caption=windowsTiming.m,language=MATLAB,numbers=left]{./windowsTiming.m}

\lstinputlisting[caption=freqDist.m,language=MATLAB,numbers=left]{./freqDist.m}
\lstinputlisting[caption=centroidSpread.m,language=MATLAB,numbers=left]{./centroidSpread.m}
\lstinputlisting[caption=centroidPlot.m,language=MATLAB,numbers=left]{./centroidPlot.m}

\lstinputlisting[caption=specFlat.m,language=MATLAB,numbers=left]{./specFlat.m}
\lstinputlisting[caption=specFlux.m,language=MATLAB,numbers=left]{./specFlux.m}
\lstinputlisting[caption=specPlot.m,language=MATLAB,numbers=left]{./specPlot.m}

\lstinputlisting[caption=melBank.m,language=MATLAB,numbers=left]{./melBank.m}
\lstinputlisting[caption=mfcc.m,language=MATLAB,numbers=left]{./mfcc.m}

\end{document}