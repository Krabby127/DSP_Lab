% !TEX TS-program = pdflatex
% !TEX encoding = UTF-8 Unicode

% This is a simple template for a LaTeX document using the "article" class.
% See "book", "report", "letter" for other types of document.

\documentclass{article} % use larger type; default would be 10pt

\usepackage[utf8]{inputenc} % set input encoding (not needed with XeLaTeX)

%%% Examples of Article customizations
% These packages are optional, depending whether you want the features they provide.
% See the LaTeX Companion or other references for full information.

%%% PAGE DIMENSIONS
\usepackage{geometry} % to change the page dimensions
\geometry{letterpaper} % or letterpaper (US) or a5paper or....
 \geometry{margin=1in} % for example, change the margins to 2 inches all round
% \geometry{landscape} % set up the page for landscape
%   read geometry.pdf for detailed page layout information
\usepackage{lipsum}
%\usepackage{cases}
\usepackage{graphicx} % support the \includegraphics command and options
\usepackage{subcaption}
\usepackage[normalem]{ulem}
% \usepackage[parfill]{parskip} % Activate to begin paragraphs with an empty line rather than an indent
%\usepackage[subsection]{placeins}
\usepackage{float}
\usepackage [autostyle, english = american]{csquotes}
%%% PACKAGES
\usepackage{booktabs} % for much better looking tables
\usepackage{array} % for better arrays (eg matrices) in maths
\usepackage{paralist} % very flexible & customisable lists (eg. enumerate/itemize, etc.)
\usepackage{verbatim} % adds environment for commenting out blocks of text & for better verbatim
\usepackage{subfig} % make it possible to include more than one captioned figure/table in a single float
% These packages are all incorporated in the memoir class to one degree or another...
\usepackage{mcode}
\usepackage{xfrac}
\usepackage{lstlinebgrd}
%%% HEADERS & FOOTERS
\usepackage{fancyhdr} % This should be set AFTER setting up the page geometry
\pagestyle{fancy} % options: empty , plain , fancy
\renewcommand{\headrulewidth}{0pt} % customise the layout...
\lhead{}\chead{}\rhead{}
\lfoot{}\cfoot{\thepage}\rfoot{}
\usepackage{mathtools}
%%% SECTION TITLE APPEARANCE
\usepackage{sectsty}
\allsectionsfont{\sffamily\mdseries\upshape} % (See the fntguide.pdf for font help)
% (This matches ConTeXt defaults)

%%% ToC (table of contents) APPEARANCE
\usepackage[nottoc,notlof,notlot]{tocbibind} % Put the bibliography in the ToC
\usepackage[titles,subfigure]{tocloft} % Alter the style of the Table of Contents
\renewcommand{\cftsecfont}{\rmfamily\mdseries\upshape}
\renewcommand{\cftsecpagefont}{\rmfamily\mdseries\upshape} % No bold!
\renewcommand{\topfraction}{0.85}
\renewcommand{\textfraction}{0.1}
\renewcommand{\floatpagefraction}{0.85}

%%% END Article customizations
\usepackage{float}
\usepackage{framed}
\usepackage{amsfonts}
\usepackage{amsmath}
\usepackage{mathabx}
%%% The "real" document content comes below...
\newcommand{\includecode}[2][c]{\lstinputlisting[caption=#2, escapechar=, style=custom#1]{#2}<!---->}
\title{Lab 5}
\author{Michael Eller}
\date{April 11, 2016} % Activate to display a given date or no date (if empty),
         	      % otherwise the current date is printed 

\begin{document}
\maketitle

\tableofcontents
\clearpage
\section{Introduction}

In this lab, we will be implementing the JPEG algorithm for image compression. 
Typically, raw images can become incredibly large and it becomes difficult to store
such large images. Therefore, we use an efficient compression algorithm that
allows us to reduce the size needed to store the image, while still maintaining
high fidelity. 

\section{The JPEG Compression Algorithm}

In this section, we will show the three main steps involved in the JPEG
compression algorithm. 

\subsection{The Forward and Inverse Discrete Cosine Transform}
The original image is divided into non overlapping blocks of size $8 \times 8$.
The blocks are scanned horizontally, and then vertically. Each block is transformed
using a two dimensional discrete cosine transform,

\begin{equation}
\label{eq:dct2}
F_{u,v}=\frac{1}{4} C_u C_v \sum\limits_{x=0}^{7} \sum\limits_{y=0}^{7}
f(x,y) \cos \frac{(2x+1)u \pi}{16} \cos \frac{(2y+1)v \pi}{16}, \quad u,v=0,\dots,7
\end{equation}

where,

\begin{equation}
\label{eq:cu}
C_u=
\begin{cases}
1 & \texttt{if } u \neq 0\\
\frac{1}{\sqrt{2}} & \texttt{if } u=0
\end{cases}
\end{equation}

The coefficients $F_{u,v}$ are coarsely approximated over a small set of possible values (quantization),
and replaced by $\widetilde{F}_{u,v}$. The \emph{quantization} process cannot be inverted and contributes to the loss in image quality. \\

Given the coefficients $\widetilde{F}_{u,v}$, one can use the inverse Discrete cosine transform to reconstruct an estimate
of the block using

\begin{equation}
\label{eq:idct2}
\widetilde{f}_{x,y}=\frac{1}{4} C_u C_v \sum\limits_{u=0}^{7} \sum\limits_{v=0}^{7}
\widetilde{F}_{u,v} \cos \frac{(2x+1)u \pi}{16} \cos \frac{(2y+1)v \pi}{16}, \quad x,y=0,\dots,7
\end{equation}

%\begin{equation}
%\label{eq:KL}
%KL(G^s,G^{s'})=\frac{1}{2}\left (\text{tr}(\Sigma_{s'}^{-1}\Sigma_s) + (\mu_{s'}-\mu_s)^T \Sigma_{s'}^{-1} (\mu_{s'}-\mu_s)-K+
%\log\left( \frac{\det \Sigma_{s'}}{\det \Sigma_s} \right) \right )
%\end{equation}





\begin{framed}
\textbf{Assignment}
\begin{enumerate}
\item Write the MATLAB function \verb|dctmgr| that implements the following functionality

\begin{enumerate}
\item It takes a luminance (gray-level) image as an input, divides it into non overlapping
$8 \times 8$ blocks, and DCT transform each block according to Equation \ref{eq:dct2}.

\item The DCT coefficients for the entire image are returned into a matrix \verb|coeff| of size
$64 \times N_{blocks}$, where $N_{blocks}$ is the number of $8 \times 8$ blocks inside the image. 

\item For a given block $b$, the coefficients in the column \verb|coeff(:,b)| are organized according to
the specified zig-zag order.

\item The zero-frequency coefficients, $F(1,1)$, for each block is encoded using differential encoding. 
\end{enumerate}
\item Write the MATLAB function \verb|idctmgr| that implements the following functionality:

\begin{enumerate}
\item It takes a matrix \verb|coeff| of size $64 \times N_{blocks}$, where $N_{blocks}$ is the number of
$8 \times 8$ blocks inside the image and reconstruct a luminance image. 

\item For each given block \verb|b|, the coefficients in the column \verb|coeff(:,b)| are used to reconstruct
the block according to Equation \ref{eq:idct2}.
\end{enumerate}

\end{enumerate}
\end{framed}

\lstinputlisting[caption=QTable.m,language=MATLAB,numbers=left]
{./QTable.m}  
\lstinputlisting[caption=dctmgr.m,language=MATLAB,numbers=left]
{./dctmgr.m} 
\lstinputlisting[caption=toZigzag.m,language=MATLAB,numbers=left]
{./toZigzag.m} 
\lstinputlisting[caption=idctmgr.m,language=MATLAB,numbers=left]
{./idctmgr.m} 
\lstinputlisting[caption=invZigzag.m,language=MATLAB,numbers=left]
{./invZigzag.m} 


\subsection{Quantization and Inverse Quantization}
The effect of the DCT is to create many small coefficients that are close to zero, and a small number
of large coefficients. We can think of the DCT as a rotation of the space $\mathbb{R}^{64}$ that aligns
the coordinate system along the direction associated with the largest variance in the distribution of coefficients. 

In the JPEG compression standard, the quantization of the DCT coefficients is performed as follows

\begin{equation}
\label{eq:compr}
F_{u,v}^{q} = \texttt{round}\left( \frac{F_{u,v}}{\texttt{loss-factor} \times Q_{u,v}}\right)
\end{equation}

The inverse quantization of the DCT coefficients is defined by the map

\begin{equation}
\label{eq:decompr}
\widetilde{F}_{u,v}=F_{u,v}^{q} \times \texttt{loss-factor} \times Q_{u,v}
\end{equation}

We note that the compression becomes lossy during the quantization step performed in Equation
\ref{eq:compr} that created many zero coefficients as \verb|loss-factor| becomes larger. The
quantization step $Q_{u,v}$ varies as a function of the frequency. 


\begin{framed}
\textbf{Assignment}
\begin{enumerate}
\setcounter{enumi}{2}
\item Implement the quantizer and inverse quantizer, as defined by Equations A and B, and integrate it with the
DCT and IDCT transforms.

\item For the six test images available on D2L, display reconstructed images for \verb|loss-factor = 1|, 
\verb|loss-factor = 10|, and \verb|loss-factor = 20|. Display also the reconstruction error. 

\item For the six test images, compute the Peak Signal to Noise Ratio (PSNR) between the original
image, and the reconstructed image,

\begin{equation}
\texttt{PSNR}= 10 \log_{10} \left( \frac{255^2}{\frac{1}{N^2} \sum_{i,j=0}^{N-1} |f(i,j) - \widetilde{f}(i,j)|^2} \right)
\end{equation}

\end{enumerate}
\end{framed}

The remaining images can be found in Appendix \ref{ap:fig}. 

\begin{figure}[H]
\centering
\includegraphics[width=0.8\textwidth]{barbaraL1_dct.png}
\caption{Barbara: Loss Factor = 1}
\end{figure}

\begin{figure}[H]
\centering
\includegraphics[width=0.8\textwidth]{barbaraL10_dct.png}
\caption{Barbara: Loss Factor = 10}
\end{figure}

\begin{figure}[H]
\centering
\includegraphics[width=0.8\textwidth]{barbaraL20_dct.png}
\caption{Barbara: Loss Factor = 20}
\end{figure}

\subsection{Variable Length and Runlength Encoding}

In this section we are interested in the coding of the quantized DCT coefficients associated with frequencies that are not
zero, that is for $(u,v)\neq (0,0)$. 

In our variable length decoding/encoding, we will be using three data points: 
\begin{itemize}
\item nZeros
\item nBits
\item value
\end{itemize}
\emph{nZeros} is the number of zeros skipped sincethe last non zero coefficient. We will use 6 bits for this. 
\emph{nBits} is the number of bits needed to encode the value of the coefficient using two's complement representation.
The DC coefficient needs 12 bits and the other coefficnets need 11 bits each. 
\emph{value} is the actual value of the coefficient encoded using two's complement representation. 



\begin{framed}
\textbf{Assignment}
\begin{enumerate}
\setcounter{enumi}{5}
\item Implement the variable length coder and decoder that is used to encode the quantized coefficients. 
The input paramater should be an array \verb|quant| of 64 quantized coefficients. The output should
be an array of signed 16 bit integes, \verb|symb|, and you will store sequentially the values \verb|nZeros|,
\verb|nBits|, and \verb|value| for each non zero coefficient. The last two entries should be 00 followed
by 00. The size of symb should be variable, but cannot exceed $192 = 3 \times 64$. 
\end{enumerate}
\end{framed}


\lstinputlisting[caption=jpegEncoder.m,language=MATLAB,numbers=left]
{./jpegEncoder.m}
\lstinputlisting[caption=bin2sdec.m,language=MATLAB,numbers=left]
{./bin2sdec.m}  
\lstinputlisting[caption=jpegDecoder.m,language=MATLAB,numbers=left]
{./jpegDecoder.m}  


\begin{framed}
\textbf{Assignment}
\begin{enumerate}
\setcounter{enumi}{6}
\item Implement the arithmetic coder to encode the sequence of variable length integers that is stored in \verb|symb|. 
The input to your function should be the array \verb|symb|, and the output should be an array \verb|bitcode|. 
Each entry of \verb|bitcode| is a 0 or a 1, adn the actual size of the code is simply given by \verb|length(bitcode)|. 

\item Implement the arithmetic decoder the sequence of 0s and 1s and reconstruct the sequence of variable length
integers. The input to your function should be \verb|bitcode| and the length of \verb|symb|. The output should be an array \verb|symb|. 

\item You will now evaluate the quality of the coder. For the siz test images, you will compute the PSNR between the 
reconstructed image and the original image, for every value of the parameter \verb|loss-factor =| $1,\dots,100$. You will also
compute the true number of bits needed to compress the image. This will lead to a compression ratio, given by

\begin{equation}
\texttt{compression ratio} = \frac{8 \times \texttt{number of rows } \times \texttt{number of columns }}{\texttt{total number of bits}}
\end{equation}
You will plot a curve that gives the PSNR as a function of the compression ratio. 
\end{enumerate}
\end{framed}


During my haste to complete the encoder/decoder from the previous question, I inadvertently wrote the bitstream encoder as the solution
Assignment 6. However, upon completing my encoder, I noticed that I always seem to get the same compression ratio of 32. 
The output images for my \verb|dctmgr| also do not seem to change by much depending on the \verb|quality-factor|. 
Perhaps this bug will be fixed in a later release of Lab 5. 


\begin{figure}[H]
\centering
\includegraphics[width=0.8\textwidth]{psnr.png}
\caption{PSNR.png}
\end{figure}

\begin{figure}[H]
\centering
\includegraphics[width=0.8\textwidth]{compress.png}
\caption{compress.png}
\end{figure}



%%%%%%%%%%%%%%%%%%%%%%%%%%%%%%%%%%%%%%%%%%%%%%%%%%%%%%%%%%%%%
\clearpage
\appendix

\section{Figures}
\label{ap:fig}

\begin{figure}[H]
\centering
\includegraphics[width=0.8\textwidth]{psnr.png}
\caption{PSNR.png}
\end{figure}

\subsection{Barbara}
\begin{figure}[H]
\centering
\includegraphics[width=0.8\textwidth]{barbaraL1_dct.png}
\caption{Barbara: Loss Factor = 1}
\end{figure}

\begin{figure}[H]
\centering
\includegraphics[width=0.8\textwidth]{barbaraL10_dct.png}
\caption{Barbara: Loss Factor = 10}
\end{figure}

\begin{figure}[H]
\centering
\includegraphics[width=0.8\textwidth]{barbaraL20_dct.png}
\caption{Barbara: Loss Factor = 20}
\end{figure}

\subsection{Clown}
\begin{figure}[H]
\centering
\includegraphics[width=0.8\textwidth]{clownL1_dct.png}
\caption{clown: Loss Factor = 1}
\end{figure}

\begin{figure}[H]
\centering
\includegraphics[width=0.8\textwidth]{clownL10_dct.png}
\caption{clown: Loss Factor = 10}
\end{figure}

\begin{figure}[H]
\centering
\includegraphics[width=0.8\textwidth]{clownL20_dct.png}
\caption{clown: Loss Factor = 20}
\end{figure}

\subsection{Fingerprint}
\begin{figure}[H]
\centering
\includegraphics[width=0.8\textwidth]{fingerprintL1_dct.png}
\caption{fingerprint: Loss Factor = 1}
\end{figure}

\begin{figure}[H]
\centering
\includegraphics[width=0.8\textwidth]{fingerprintL10_dct.png}
\caption{fingerprint: Loss Factor = 10}
\end{figure}

\begin{figure}[H]
\centering
\includegraphics[width=0.8\textwidth]{fingerprintL20_dct.png}
\caption{fingerprint: Loss Factor = 20}
\end{figure}

\subsection{mandril}
\begin{figure}[H]
\centering
\includegraphics[width=0.8\textwidth]{mandrilL1_dct.png}
\caption{mandril: Loss Factor = 1}
\end{figure}

\begin{figure}[H]
\centering
\includegraphics[width=0.8\textwidth]{mandrilL10_dct.png}
\caption{mandril: Loss Factor = 10}
\end{figure}

\begin{figure}[H]
\centering
\includegraphics[width=0.8\textwidth]{mandrilL20_dct.png}
\caption{mandril: Loss Factor = 20}
\end{figure}

\subsection{Roof}
\begin{figure}[H]
\centering
\includegraphics[width=0.8\textwidth]{roofL1_dct.png}
\caption{roof: Loss Factor = 1}
\end{figure}

\begin{figure}[H]
\centering
\includegraphics[width=0.8\textwidth]{roofL10_dct.png}
\caption{roof: Loss Factor = 10}
\end{figure}

\begin{figure}[H]
\centering
\includegraphics[width=0.8\textwidth]{roofL20_dct.png}
\caption{roof: Loss Factor = 20}
\end{figure}

\subsection{Straw}
\begin{figure}[H]
\centering
\includegraphics[width=0.8\textwidth]{strawL1_dct.png}
\caption{straw: Loss Factor = 1}
\end{figure}

\begin{figure}[H]
\centering
\includegraphics[width=0.8\textwidth]{strawL10_dct.png}
\caption{straw: Loss Factor = 10}
\end{figure}

\begin{figure}[H]
\centering
\includegraphics[width=0.8\textwidth]{strawL20_dct.png}
\caption{straw: Loss Factor = 20}
\end{figure}


\section{Code}
\lstinputlisting[caption=buildReport.m,language=MATLAB,numbers=left]
{./buildReport.m} 
\lstinputlisting[caption=dctmgr.m,language=MATLAB,numbers=left]
{./dctmgr.m} 
\lstinputlisting[caption=toZigzag.m,language=MATLAB,numbers=left]
{./toZigzag.m} 
\lstinputlisting[caption=idctmgr.m,language=MATLAB,numbers=left]
{./idctmgr.m} 
\lstinputlisting[caption=invZigzag.m,language=MATLAB,numbers=left]
{./invZigzag.m} 
\lstinputlisting[caption=QTable.m,language=MATLAB,numbers=left]
{./QTable.m}  
\lstinputlisting[caption=ParforProgressMonitor.java,language=JAVA,numbers=left]
{./java/ParforProgressMonitor.java} 
\lstinputlisting[caption=jpegEncoder.m,language=MATLAB,numbers=left]
{./jpegEncoder.m}
\lstinputlisting[caption=bin2sdec.m,language=MATLAB,numbers=left]
{./bin2sdec.m}  
\lstinputlisting[caption=jpegDecoder.m,language=MATLAB,numbers=left]
{./jpegDecoder.m}  
\lstinputlisting[caption=plot3Images.m,language=MATLAB,numbers=left]
{./plot3Images.m}  

\end{document}